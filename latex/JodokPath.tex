%% Generated by Sphinx.
\def\sphinxdocclass{book}
\IfFileExists{luatex85.sty}
 {\RequirePackage{luatex85}}
 {\ifdefined\luatexversion\ifnum\luatexversion>84\relax
  \PackageError{sphinx}
  {** With this LuaTeX (\the\luatexversion),Sphinx requires luatex85.sty **}
  {** Add the LaTeX package luatex85 to your TeX installation, and try again **}
  \endinput\fi\fi}
\documentclass[a5paper,10pt,english]{book}
\ifdefined\pdfpxdimen
   \let\sphinxpxdimen\pdfpxdimen\else\newdimen\sphinxpxdimen
\fi \sphinxpxdimen=.75bp\relax
\ifdefined\pdfimageresolution
    \pdfimageresolution= \numexpr \dimexpr1in\relax/\sphinxpxdimen\relax
\fi
%% let collapsible pdf bookmarks panel have high depth per default
\PassOptionsToPackage{bookmarksdepth=5}{hyperref}
%% turn off hyperref patch of \index as sphinx.xdy xindy module takes care of
%% suitable \hyperpage mark-up, working around hyperref-xindy incompatibility
\PassOptionsToPackage{hyperindex=false}{hyperref}
%% memoir class requires extra handling
\makeatletter\@ifclassloaded{memoir}
{\ifdefined\memhyperindexfalse\memhyperindexfalse\fi}{}\makeatother

\PassOptionsToPackage{booktabs}{sphinx}
\PassOptionsToPackage{colorrows}{sphinx}

\PassOptionsToPackage{warn}{textcomp}

\catcode`^^^^00a0\active\protected\def^^^^00a0{\leavevmode\nobreak\ }
\usepackage{cmap}
\usepackage{fontspec}
\defaultfontfeatures[\rmfamily,\sffamily,\ttfamily]{}
\usepackage{amsmath,amssymb,amstext}
\usepackage{polyglossia}
\setmainlanguage{english}


\usepackage{fontspec}\setmainfont{TeX Gyre Pagella}



\usepackage{sphinx}
\sphinxsetup{
        HeaderFamily=\bfseries,
        TitleColor={rgb}{0,0,0},
        InnerLinkColor={rgb}{0,0,0},
        hmargin={1.5cm,2cm},
        vmargin={2cm,2cm},
    }
\fvset{fontsize=auto}
\usepackage{geometry}


% Include hyperref last.
\usepackage{hyperref}
% Fix anchor placement for figures with captions.
\usepackage{hypcap}% it must be loaded after hyperref.
% Set up styles of URL: it should be placed after hyperref.
\urlstyle{same}


\usepackage{sphinxmessages}



        \usepackage{emptypage}
        \usepackage{titling}
        \makeatletter
        \fancypagestyle{normal}{
          \fancyhf{}
          \fancyfoot[LE,RO]{\thepage}
          \fancyfoot[RE,LO]{}
          \fancyhead[RE]{\releasename}
          \fancyhead[LE]{\@title}
          \fancyhead[RO]{\emph{\leftmark}}
          \renewcommand{\headrulewidth}{0.4pt}
          \renewcommand{\footrulewidth}{0pt}
        }
        \fancypagestyle{plain}{
          \renewcommand{\footrulewidth}{0pt}
          \fancyhead{}
          \renewcommand{\headrulewidth}{0pt}
        }
        \makeatother
        \setotherlanguage{thai}
        \newfontfamily\thaifont[Script=Thai]{Norasi}
        \newcommand{\DUrolethai}[1]{\textthai{#1}}
        % \renewcommand{\chaptermark}[1]{\markboth{#1}{}}
        \usepackage[numbered]{bookmark}
        \iffalse  \let\frontmatter\relax \let\mainmatter\relax \let\backmatter\relax \fi
    

\title{The Path to Nibbāna}
\date{Nov 19, 2024}
\release{b066528}
\author{Phra Dhamma Theerarach Mahamuni (Jodok)}
\newcommand{\sphinxlogo}{\vbox{}}
\renewcommand{\releasename}{b066528}
\makeindex
\begin{document}

\pagestyle{empty}

\makeatletter%
\hypersetup{pdfauthor={\@author}, pdftitle={\@title}}%
\makeatother%
\begin{titlepage}%
    \vspace*{\baselineskip}
    \vfill
    \hbox{%
        \hspace*{0.15\textwidth}%
        \rule{1pt}{.95\textheight}
        \hspace*{0.05\textwidth}%
        \parbox[b]{0.8\textwidth}{
            \vbox to.95\textheight{%
                \vspace{.05\textheight}
                {\noindent\Huge\bfseries The Path\\[0.5\baselineskip]
                to Nibbāna}\\[4\baselineskip]
                {\Large\emph{Phra Dhamma Theerarach Mahamuni \\ (Ajahn Jodok)}}\par
                \vfill % space{0.3\textheight}
                Other formats (PDF, HTML, ePub, …) available from \href{https://github.com/edhamma/jodok-path}{github.com/edhamma/jodok-path}.
                \\[\baselineskip]
                {\noindent This e-book is a community effort. If you spot an error in the text (such as misspelled word), go to the address above and report or (if you have the skill) fix it. Thanks!}
                \\[\baselineskip]
                {\noindent Revision \releasename, built \today.}
            }% end of vbox
        }% end of parbox
    }% end of hbox
    \vfill
\end{titlepage}

\pagestyle{plain}
\sphinxtableofcontents
\pagestyle{normal}
\phantomsection\label{\detokenize{index::doc}}


\frontmatter
\bgroup
\def\thesection{\arabic{section} }

\sphinxstepscope


\chapter{Front}
\label{\detokenize{front:front}}\label{\detokenize{front::doc}}
\begin{figure}[htbp]
\centering
\capstart

\noindent\sphinxincludegraphics{{portrait}.jpg}
\caption{The Ven. Phra Dhamma Theerarach Mahamuni}\label{\detokenize{front:id3}}
\begin{sphinxlegend}
\sphinxAtStartPar
\DUrole{thai}{พระธรรมธีรราชมหามุนี (โชดก ญาณสิทฺธิเถร ส.ธ.๙)}

\sphinxAtStartPar
\DUrole{thai}{พระอาจารย์าใหญ่ฝ่ายวิปัสสนาธุระ}
\end{sphinxlegend}
\end{figure}

\setcounter{section}{0}

\begin{DUlineblock}{0em}
\item[] THE PATH TO NIBBĀNA
\item[] (An Introduction to Insight Meditation)
\item[] by The Ven.Phra Dhamma Theerarach Mahamuni (Jodok)
\item[] Graduate of the Highest State Pali Examination of Thailand.
\item[] Chief Master for Vipassana Meditation in Thailand.
\item[] 
\item[] Vipassana Center Section 5, Mahādhātu Monastery
\item[] Bangkok 10200 Thailand.
\item[] Tel. 2226011
\item[] 
\item[] NINTH EDITION 2014
\end{DUlineblock}

\begin{sphinxadmonition}{note}{Electronic edition notice}

\sphinxAtStartPar
In a few places, obvious erros of the print were corrected. Footnotes with Pali locations were added. Spelling of Pali words was left intact.
\end{sphinxadmonition}

\clearpage


\section{Foreword to the English Translation}
\label{\detokenize{front:foreword-to-the-english-translation}}
\sphinxAtStartPar
The path to Nirvana, which you are holding in your hands, is translated from “The Path to Nirvava\sphinxhyphen{} Thai version \sphinxhyphen{}. It is a directt ranaslation of the original book of the Ven. Phra Dhamma Threerarach Maharnuni (Jodok Yannasit). The content of this book is emphasized to the Insight Vipassana method handbook,including the evaluation of meditation as The Discourse of Four Foundations of Mindfulness and the result of insight meditation. The path to Nirvana is suitable for all people who would like to understand, get inside and develop their mind. The wisdom and peace from meditation will encourage family, society, nation, religion and then make the peaceful world.

\sphinxAtStartPar
This book has been published in 8 editions, during various occasions. This edition is the 9th, the English publishing for each edition is on behalf of the burden of contemplation division, Vipassana Center, where the Center of Buddhist propagation. This international Vipassana Center has been supported by The Ven. Phra Dhamma Threerarach Mahamuni (Jodok Yannasit), who is the former abbot of Mahadhatu Monsatery, and the Ven. Somdet buddahchan (Art Asapathera).

\begin{DUlineblock}{0em}
\item[] Phrakhruvimolthammarangsee
\item[] Vice\sphinxhyphen{}abbot of Mahadhatu Temple
\item[] Section 5
\end{DUlineblock}

\clearpage


\section{Translator’s Introduction}
\label{\detokenize{front:translator-s-introduction}}
\sphinxAtStartPar
This book is the work of the late Ven. Phra Dhamma Theerarach Mahamuni Mahāthera, the late head of the meditation masters in Thailand, who passed away on the 30th of June 1988.

\sphinxAtStartPar
While he was alive he taught meditation to both Thais and foreigners in Thailand and throughout the world. The Vipassana Centre at Wat Mahadhatu considers that “the Path to Nibbana” is one of the most useful of his books for meditators and those who are interested in meditation. This book covers both theory and practical exercises. So the Vipassana Centre at Section 5, Wat Mahadhatu has decided to reprint this book, to be used as a guide to help those who are interested in Vipassana meditation.

\sphinxAtStartPar
There are three parts to this book. Part 1 is theory. Part 2 is exercises for meditation practice and part 3 is a manual for checking you Vipassana progress.

\sphinxAtStartPar
We hope that this book will prove useful for those who are interested in practising meditation. It can be used as a guide for beginners, but the book alone is not enough, it should be used as additional guidance when practising with a meditation teacher. Those who have extensive experience should consult a meditation master.

\sphinxAtStartPar
We sincerely hope you make progress in your meditation practice.

\begin{DUlineblock}{0em}
\item[] May all beings be well and happy.
\item[] \sphinxstylestrong{Vorasak Jandamit}
\item[] \sphinxstylestrong{Helen Jandamit}
\item[] on behalf of the Vipassana Centre at Wat Mahadhatu, Bangkok.
\item[] 1st November, 1989
\end{DUlineblock}

\clearpage


\section{Foreword to the 3rd Edition}
\label{\detokenize{front:foreword-to-the-3rd-edition}}\begin{quote}

\sphinxAtStartPar
\sphinxstyleemphasis{Dhammo have rakkhati dhaṃmacāriṃ chattaṃ mahanthaṃ viya vassakāle.} %
\begin{footnote}[1]\sphinxAtStartFootnote
Nettippakaraṇapāḷi, Paṭi­niddesa­vāra, Vibhaṅga 6 (\sphinxhref{https://dhammatalks.net/suttacentral/sc2016/sc/pi/ne9.html}{online}).
%
\end{footnote}

\sphinxAtStartPar
The Dhamma shelters the Dhamma\sphinxhyphen{}followers like a Great umbrella in the rainy season.
\end{quote}

\sphinxAtStartPar
The Principal Teaching of Lord Buddha comprises three categories: The Study of the Scriptures (Pariyatti Dhamma), the Practice of the Dhamma (Patipatti\sphinxhyphen{}Dhamma), and Realization (Pativedha\sphinxhyphen{}Dhamma). They depend upon each other. Then they can develop Buddhism in the future.

\sphinxAtStartPar
\sphinxstylestrong{The Study of the Scriptures} refers to the study of the Tipitaka, the Three Collections of the Buddha’s Teaching in which are contained morality (Sila), concentration (Samadhti) and Wisdom (Paññā).

\sphinxAtStartPar
\sphinxstylestrong{The Practice of the Dhamma} is directed towards training in and development of ethical conduct, concentration of mind and intuitive wisdom through the system of Budhist Meditation.

\sphinxAtStartPar
\sphinxstylestrong{The Stage of Realization being} the result of the practice, brings about Enlightenment and Complete Freedom from all forms of mental defilements. This is termed, “Realization” according to the Buddhist sense and aim of life.

\sphinxAtStartPar
The Study of the Scriptures is like a whole coconut.

\sphinxAtStartPar
The Practice of the Dhamma is like breaking a coconut.

\sphinxAtStartPar
The stage of Realizations is like breaking a coconut and eating all its contents.

\sphinxAtStartPar
All followers should cultivate these three stages so that they will have peace and happiness in present and future lives.

\sphinxAtStartPar
May they all attain the happiness of Nibbana.

\begin{DUlineblock}{0em}
\item[] \sphinxstylestrong{The Ven. Phra Dhamma Theerarach Mahamuni}
\item[] Vipassana Meditation Centre,
\item[] Section 5, Mahadhatu Monastery
\item[] 9 October 1971
\end{DUlineblock}

\egroup
\mainmatter
% promote sections for the main text
% (unlike in frontmatter and appendix)
\iffalse
   \let\subsubsection\subsection
   \let\subsection\section
   \let\section\chapter
   \let\chapter\part
\fi

\sphinxstepscope


\chapter{The Path}
\label{\detokenize{path:the-path}}\label{\detokenize{path::doc}}

\section{(Path)}
\label{\detokenize{path:path}}
\sphinxAtStartPar
In Buddhist tradition the term “Path” has two senses, one being “Pakati maggo” or an ordinary path, i.e. a byway for men and animals, and another “Patipadā maggo” or the path of good or bad behaviour for men alone, traversed through deeds, words and thoughts.

\sphinxAtStartPar
Patipadā maggo is divided into five kinds:
\begin{enumerate}
\sphinxsetlistlabels{\arabic}{enumi}{enumii}{}{.}%
\item {} 
\sphinxAtStartPar
The Descending Path, brought about by offences against the normal Code, and based on Greed, Hatred and Delusion.

\item {} 
\sphinxAtStartPar
The Human path, the path of five moralities or the 10\sphinxhyphen{}fold wholesome course of action (Kusala\sphinxhyphen{}Kamma\sphinxhyphen{}Patha).

\item {} 
\sphinxAtStartPar
The Path to the Six Classes of Heaven, which comprises eight classes of moral consciousness, culminating in Moral Shame (Hiri) and Moral Dread (Ottappa), resulting in alms\sphinxhyphen{}giving, attending sermons, building chapels, temples, ecclesiastical schools, hospitals and ordinary schools.

\item {} 
\sphinxAtStartPar
The Path to the Abode of Brahma which is the development of tranquility of mind (Samatha bhavana) by means of meditation upon any of the forty traditional subjects; very briefly, these are classified technically as the ten “Kasina” (Contemplation devices), ten “Asubhas” (Impurities), ten Anussatis“ (Reflections), four ”Brahma\sphinxhyphen{}Vihāras“ (Sublime States), one ”Ahārepatikūlasaññ a“ (Reflection one the loathsomeness of food), one ”Catudhātu Vavatthāna“ (Analysis of the four elements), and four ”Arupakammatthāna“ (Stages of arūpa\sphinxhyphen{}jhāna).

\item {} 
\sphinxAtStartPar
The Path to Nibbāna, Pali; (Sanskrit:Nirvāna), which is the development of Insight (Vipassana bhavana), having Nāmarūpa, or mental and physical states, as the objects of meditation. Of these five paths, the fifth is the one under consideration and is known as “ekāyana magga” (The Only Way). It is so called because of the following qualities:
\begin{enumerate}
\sphinxsetlistlabels{\arabic}{enumii}{enumiii}{}{.}%
\item {} 
\sphinxAtStartPar
It is a straight path, never branching into byways.

\item {} 
\sphinxAtStartPar
It is the path for a person who leaves society and retires into a secluded place to practise.

\item {} 
\sphinxAtStartPar
It is the path of the Buddha himself, as He was the discoverer of this path by his own effort.

\item {} 
\sphinxAtStartPar
It is a path found in Buddhism only, and not explicitly in other religions.

\item {} 
\sphinxAtStartPar
It is the path leading to Nibbāna, as set out in the Pali Scriptures:

\end{enumerate}

\end{enumerate}

\sphinxAtStartPar
\sphinxstyleemphasis{“Cattārome bhikkhave satipatthānā bhavitā bahulikata ekantanibbitāya virāgāya nirothāya upasamāya abhiññaya sambodhaya nibbanāya samvattati”}
\begin{quote}

\sphinxAtStartPar
“Brethren! These four Foundations of Mindfulness, (Satipatthānā), when fully practised, produce detachment, freedom from craving, complete release, perfect bliss, perfect wisdom, enlightenment, Nibbāna.”
\end{quote}

\sphinxAtStartPar
\sphinxstyleemphasis{“Seyyathāpi bhikkhave gamganadipācinaninnā pācinaponā pācinapabbharā evameva kho bhikkhave cattāro satipatthāne bhavento satipatthāne bahulikarōnto nibbānaninno hoti nibbānapono nibbānapabbhāro.”}
\begin{quote}

\sphinxAtStartPar
“Brethren! As the Ganges flows, rushes and races towards the West, so does a bhikkhu who develops and practises the Four Foundations of Mindfulness tend towards Nibbāna.”
\end{quote}


\section{Questions and Answers}
\label{\detokenize{path:questions-and-answers}}
\sphinxAtStartPar
\sphinxstylestrong{Q.} Where and when do the five Aggregates (Khandha) of the present moment, which are reducible to matter and mind (rūpa dhamma and nama dhamma) arise and cease?

\sphinxAtStartPar
\sphinxstylestrong{A.} They arise at the six internal sense\sphinxhyphen{}bases (āyatana), namely: eyes, ears, nose, tongue, body and mind\sphinxhyphen{}base, and at the six external sense\sphinxhyphen{}bases : visible object, sound, odour, taste, body\sphinxhyphen{}impression and mind object, whenever one’s eyes see a form, ears hear a sound, nose senses a smell, tongue experiences a taste, body contacts something cold, hot, soft, or hard, or one’s mind seizes upon an idea; and they cease whence they arise, being born and perishing instantly.

\sphinxAtStartPar
\sphinxstylestrong{Q.} Whence and when do greed, hatred and delusion arise and cease?

\sphinxAtStartPar
\sphinxstylestrong{A.} They also arise at the internal and external sense\sphinxhyphen{}bases, for example, when one’s eyes see a form, grasping at it is greed (lobha) and hatred (dosa)“ lack of mindfulness in acknowledging the reality behind the form is unawareness. The same applies to the other kinds of perception.

\sphinxAtStartPar
\sphinxstylestrong{Q.} While greed, hatred and delusion can still arise, can Man be free from the Descending Path?

\sphinxAtStartPar
\sphinxstylestrong{A.} He cannot.

\sphinxAtStartPar
\sphinxstylestrong{Q.} If that is so, what can one do to avoid the Descending Path?

\sphinxAtStartPar
\sphinxstylestrong{A.} One has to carry out the path to Nibbāna.

\sphinxAtStartPar
\sphinxstylestrong{Q.} What is the path to Nibbāna?

\sphinxAtStartPar
\sphinxstylestrong{A.} It is based on the Four Foundations of Mindfulness by means of Insight Meditation (Vipassanā Bhavanā):
\begin{itemize}
\item {} 
\sphinxAtStartPar
Contemplation of the Body (Kāyānupassana),

\item {} 
\sphinxAtStartPar
Contemplation of Feelings (Vedanānupassanā),

\item {} 
\sphinxAtStartPar
Contemplation  of   States   of   Consciousness (Cittanupassanā),

\item {} 
\sphinxAtStartPar
Contemplation of Mind\sphinxhyphen{}objects (Dhammānupassanā).

\end{itemize}

\sphinxAtStartPar
During the Buddha’s Ministry, while the Exalted One stayed at a village called Kammasadamma in the state of Kuru, He mentioned the people of Kuru as the inspirers of His talk and thereupon gave a sermon on the practice of the Foundations of Mindfulness, which will be condensed below.

\sphinxAtStartPar
(The people of Kuru state, no matter whether they were Bhikkhus, Bhikkhunis or lay disciples were of quick understanding as their environment and social conditions were good, and all of them possessed healthy bodies and minds suitable for deep Contemplation. Knowing this, and having a gathering of the Kuru people before Him. Lord Buddha at once delivered the profound sermon, on the Four Foundations of Mindfulness, to them.)

\sphinxAtStartPar
The Kuru followers of Buddhism used to practise the Foundations of Mindfulness regularly. Even the slave labourers talked to one another on the Foundations of Mindfulness. At the waterfronts or the workshops, or wherever else they happened to be, they all discussed this subject. Whenever any person was asked which one of the four Foundations he has been practising, if he answered he had not, the others would reprove him by saying’ that, though he was alive he was really behaving as if dead. They would exhort him not to be negligent and instruct him to practise one of the Foundations of Mindfulness. Should the person addressed answer that he had been practising one of the Foundations of Mindfulness, the Kuru people would praise him three times with the words, “That is good” and commend him for living an excellent life and achieving conduct worthy of a human being, the Buddha having been bom into the world for the benefit of such as he. Considering that we have encountered Buddhism, it is therefore appropriate that all of us should practise the Dhammas of liberation without letting the time go by uselessly. If we have done great merit in the past, we shall be able to obtain Path and Fruition according to our latent disposition, as set out in the following:

\sphinxAtStartPar
\sphinxstyleemphasis{“Ekadhammo bhikkhave bhāvito bahulikato Sotāpattiphalasacchikiriyāya samvattati sakadāgamiphalasac\sphinxhyphen{}chikiriyāya samvattati,”}
\begin{quote}

\sphinxAtStartPar
meaning, “Brethren! This incomparable Dhamma, practised fully by any person, would result in Sotāpattiphala, Skādagāmiphala, Anāgāmiphala, and Arahattaphala. What is this incomparable Dhamma? It is Kāyagatāsati, Mindfulness of the Body.”
\end{quote}

\sphinxAtStartPar
\sphinxstyleemphasis{“Amatante bhikkahave na paribhuñjanti ye Kāyagatasatim na paribhuñjanti amatante bhikkhave paribhuñjanti ye Kāyagatasatim paribhuñjanti,”}
\begin{quote}

\sphinxAtStartPar
“Brethren! Those who do not practise mindfulness of the body, will never taste immortality; those who so practise mindfulness of the body, will certainly enter into immortality.”
\end{quote}

\sphinxAtStartPar
\sphinxstylestrong{Q.} Are there any preparations which the practitioner should make beforehand?

\sphinxAtStartPar
\sphinxstylestrong{A.} Certain necessary conditions are :
\begin{enumerate}
\sphinxsetlistlabels{\Alph}{enumi}{enumii}{}{.}%
\item {} 
\sphinxAtStartPar
To live near a capable instructor.

\item {} 
\sphinxAtStartPar
To keep the six guiding\sphinxhyphen{}faculties (indriya) healthy.

\item {} 
\sphinxAtStartPar
To keep the mind fixed upon the Four Foundations.

\end{enumerate}

\sphinxAtStartPar
Duties to be performed by the practitioner :\sphinxhyphen{}
\begin{enumerate}
\sphinxsetlistlabels{\Alph}{enumi}{enumii}{}{.}%
\item {} 
\sphinxAtStartPar
Make a positive resolution that one will not be discouraged as long as one has not yet attained the exalted Dhamma through great effort, diligence and perseverance.

\item {} 
\sphinxAtStartPar
Eat less, sleep less and speak less, but practise more.

\item {} 
\sphinxAtStartPar
Control one’s eyes, ears, nose, tongue, body and mind

\item {} 
\sphinxAtStartPar
Perform all actioas slowly and with constant awareness.

\item {} 
\sphinxAtStartPar
Perform all actions under the guidance of the following three healthy mental components, energy, mindfulness and awareness. The practitioner should endeavour to walk mindfully and to acknowledge the various perceptions without wishing to discontinue. This is the arousing of energy. Acknowledge every movement beforehand. This is the practice of mindfulness. When performing even the least of actions be conscious of every movement. This is the development of awareness.

\end{enumerate}

\sphinxAtStartPar
Activities to be avoided by the practitioner.
\begin{enumerate}
\sphinxsetlistlabels{\Alph}{enumi}{enumii}{}{.}%
\item {} 
\sphinxAtStartPar
Busying oneself with various jobs, such as cleaning, writing, and reading.

\item {} 
\sphinxAtStartPar
Indulging in much sleep with consequent loss of effort. The practitioner should sleep at the most four hours a day.

\item {} 
\sphinxAtStartPar
Indulging in talking and searching after friends, thus losing one’s practice of mindfulness.

\item {} 
\sphinxAtStartPar
Seeking company.

\item {} 
\sphinxAtStartPar
Lacking restraint of the senses.

\item {} 
\sphinxAtStartPar
Immoderation in eating. The proper course is to stop eating when five more mouthfuls would prove sufficient.

\item {} 
\sphinxAtStartPar
Failing to acknowledge mental activity when the mind seizes on or loses hold of an idea.

\end{enumerate}

\sphinxAtStartPar
When the mind is concentrated :

\sphinxAtStartPar
Walk mindfully for one hour, then sit down and acknowledge the various trends of body and mind as they arise, increasing the time for this practice form thirty minutes to one hour or more according to one’s capability.

\sphinxAtStartPar
Here is must be said in warning that if the energy exerted is great while the concentration is insufficient, distraction will arise. For example, when one acknowledges one’s awareness, “Rising,” “falling,” “sitting,” “touching,” if one cannot acknowledge the activity at that moment and yet continue, the energy exerted to try to do so will be too great and distraction will arise.

\sphinxAtStartPar
If concentration is too strong while energy is insufficient, apathy and weariness will ensue.

\sphinxAtStartPar
If faith is too great while reason is weak, greed will seize hold of the mind.

\sphinxAtStartPar
If reason is too strong while faith is insufficient, doubt and delusion will result.

\sphinxAtStartPar
And so the practitioner must learn through the practice of mindfulness how to bring about a balance of faith, reason, energy and concentration.

\sphinxAtStartPar
This is the way to bring these faculties or indriyas inbalance:
\begin{enumerate}
\sphinxsetlistlabels{\alph}{enumi}{enumii}{}{.}%
\item {} 
\sphinxAtStartPar
While practising the walking exercise, do it slowly and acknowledge the various movements at every moment. Your gaze should be about 4 feet in front of you, when looking down however, pain may arise at the back of the neck. If that occurs fix your gaze at a point about two metres in front of your feet. In so doing, one will not lose control of one’s mind and will also attain good concentration in the sitting posture. Truth will then be revealed when the mind has spent a certain period in deep concentration.

\item {} 
\sphinxAtStartPar
After the performance of the mindful walking, begin to acknowledge the rise and fall of the abdomen in the sitting posture. In doing so, do not restrain the mind and body too much or use too much effort. For example, there is a form of over\sphinxhyphen{}exertion which arises when one feels sleepy and tries to keep awake, or when one cannot acknowledge the constant changes in one’s mind and body, but still keeps up the effort. One should also never be too slack in practice and allow the mind to act under the sway of various unhealthy tendencies whenever it has the inclination. One should practise according to one’s capacity without too much restraint or effort and without yielding to the power of latent tendencies. This is the Path of Moderation.

\end{enumerate}

\sphinxAtStartPar
Keep one’s mindfulness constant; for example, after performing the mindful walk, acknowledge in the sitting posture every activity of the body and mind without letting mindfulness slip. Do this slowly and without agitation.

\sphinxAtStartPar
Preliminary arrangements and how to begin the practice of Insight Meditation (Vipassana).
\begin{enumerate}
\sphinxsetlistlabels{\arabic}{enumi}{enumii}{(}{)}%
\item {} 
\sphinxAtStartPar
The monks should make confession first, while the lay disciples should ask for the precepts before practice. Most of them observe the eight precepts.

\item {} 
\sphinxAtStartPar
Pay homage to the Triple Gem and one’s instructor thus: \sphinxstyleemphasis{“Imaham Bhagavā attabhavam Tumhākam pariccajami”} “Master, May I pay you homage for the purpose of practising insight meditation (Vipassanā) from this moment!”

\item {} 
\sphinxAtStartPar
Ask for the exercises as follows: \sphinxstyleemphasis{“Nibbānassa me bhante sacchikaranatthaya Kammatthānam dehi”} “Master, Will you give me instruction for in sight meditation (Vipassana) so that I may comprehend the Path, the Fruition and Nibbāna later?”

\item {} 
\sphinxAtStartPar
Extend your friendship to all beings in some such way as this:

\end{enumerate}
\begin{quote}

\sphinxAtStartPar
“May I and all beings be happy, free from suffering, free from longing for revenge, free from troubles, difficulties and dangers and be protected from all misfortune. May no being suffer loss! All beings have their own Kamma, have Kamma as origin; have Kamma as heredity; have Kamma as refuge; whatever Kamma one performs be it good or bad, returns to one.
\end{quote}
\begin{enumerate}
\sphinxsetlistlabels{\arabic}{enumi}{enumii}{(}{)}%
\setcounter{enumi}{4}
\item {} 
\sphinxAtStartPar
Practise the exercise of mindfulness of death thus: Our lives are transient and death is certain. That being so, we are fortunate to have entered upon the practice of insight meditation (Vipassanā) on this occasion as now we have not been born in vain and have not missed the opportunity to practise the Dhamma.

\item {} 
\sphinxAtStartPar
To the Buddha and his disciples, take a vow, as follows: “The path which all Buddhas, their venerable left\sphinxhyphen{}hand and right\sphinxhyphen{}hand disciples.their eighty great disciples and their Arahat disciples have taken to Nibbana, the path which is known as the Four Foundations of Mindfulness and is the path comprehended by the wise, I solemnly promise that I will follow in sincerity to attain that Path, the Fruition, and Nibbāana, according to my own initiative from this occasion onwards.”

\item {} 
\sphinxAtStartPar
“May I offer the Buddha this practice of Dhamma worthy of Dhamma!”

\item {} 
\sphinxAtStartPar
“I am certain to cross over suffering from birth, suffering from decay, suffering from diseases and suffering from death by this practice.

\item {} 
\sphinxAtStartPar
The instructor then gives advice to those beginning the practice as he sees fit.

\end{enumerate}


\section{Advice to the Practitioner}
\label{\detokenize{path:advice-to-the-practitioner}}
\sphinxAtStartPar
Now that we have been fortunate enough to meet the Dhamma, the doctrine of the Buddha, it is most appropriate to cultivate the precepts, concentration, and insight in one’s own self to the point of perfection. Those perfect in the precepts are certain to achieve happiness in the present and future lives; nevertheless these precepts are mundane (lokiya\sphinxhyphen{}silas) and it is not guaranteed that they can help people to be absolutely free from the descending path. As a result we have to cultivate precepts leading to the supra\sphinxhyphen{}mundane, to greater perfection. These are precepts for the attainment of the Path and Fruition. If we practise the exercise up to the precepts for attainment of Fruition, we are certain to be free from the Descending Path. It is, therefore, very advantageous to cultivate the precepts for the Path and Fruition in this life. If we practise the exercise with great carefulness we will be successful, but if we ignore the opportunity to practise, we cannot attain to freedom. On this occasion there still remains an opportunity for the demerits latent from the previous life to become effective, and demerits not yet performed are liable to be performed. Those already performed are liable to accumulate. Every human life is to be considered as having been bom of merits and as a wonderful opportunity.

\sphinxAtStartPar
To enter the practice of Insight Meditation (Vipassana) means the cultivation of such potentialities as perfection and the development of the Precepts, Concentration and Wisdom from the lower to the higher levels. As we have already seen, the precepts fall into two classes: mundane, consisting of the ordinary moral codes for laymen and bhikkhus, and supra\sphinxhyphen{}mundane, which are developed only in persons practising Insight Meditation (Vipassana) up to attainment of the Path. Concentration also falls into the same two classes, the mundane concentration of practitioners who have not yet attained the levels of Path and Fruition, and supra\sphinxhyphen{}mundane concentration, arising in persons practising Insight Meditation (Vipassana) up to these levels. The same applies to the faculty of Wisdom, mundane wisdom comprising insight into what is meritorious or demeritorious, beneficial or detrimental, profitable or unprofitable, and some understanding of the nature of mental and physical states and the “three characteristics” while supra\sphinxhyphen{}mundane or developed wisdom arises in those practising Insight Meditation (Vipassana) up to the levels of the Path, the Fruition and Nibbana.

\sphinxAtStartPar
Those who practise Insight Meditation (Vipassana) do so in order to learn to live a holy disciplined life to the point of perfection. It is, therefore, to be condidered our great advantage, but those who let the opportunity slip by, will realise afterwards that they met with Dhamma in letter only and missed the spirit. However, those who have practised and acquired the eyes of Wisdom will be greatly rewarded and are to be thought of as having paid the only real form of homage to the Buddha. And they are also to be condidered the true disciples of the victorious One, as may be seen from the following quotation: \sphinxstyleemphasis{“Bhikkhave mayi sasenho tissasadito va hotu”} meaning “Brethren! whoever has love for me, let him be like Tissa. Not those who offer me flowers, incense, candles and all kinds of perfumes are to be considered as having paid me true homage, but those who have practised the Dhamma worthy of Dhamma,”

\sphinxAtStartPar
Also, those who have practised Insight Meditation are to be reckoned as having furthered the cause of Dhamma as set out in Pali:
\begin{quote}

\sphinxAtStartPar
“Yāva hi ima catasso parisā mam imāya patipatti\sphinxhyphen{}pūjāya pūjessanti”, meaning, “For however long the four assemblies pay me homage with this practice of Dhamma, just so song will my religion endure, as even the full moon hangs suspended conspicuously amidst the sky at night”
\end{quote}

\sphinxAtStartPar
Those deserving persons who have joined in the practice should be considered as having done enormous benefit to themselves and to others, even including the nation, the faith, the King and the Constitution.

\sphinxAtStartPar
\sphinxstyleemphasis{“Vuddhim virulhim vepullam pappotu Buddhasā sane.”} “Finally may you all be prosperous, flourishing and fortunate in Dhamma; in other words, may you attain the Path, the Fruition and Nibbhāna.”

\sphinxAtStartPar
Having given such advice, the Instructor should being to give exercises to those entering the practice, thus:
\begin{enumerate}
\sphinxsetlistlabels{\alph}{enumi}{enumii}{}{.}%
\item {} 
\sphinxAtStartPar
Instruct them to walk mindfully, and to acknowledge the movements in their mind in some such way as “Right moves thus, Left moves thus.” Teach them also to be mindful while standing and turning around.

\item {} 
\sphinxAtStartPar
Instruct them how to concentrate in the sitting position, i.e. to meditate on the rising and falling of the abdomen, acknowledging the movement: “Rising, Falling,” and teach them how to recline in a posture suitable for concentration. (Note of the translator: In meditating on the rising and falling of the abdomen one has to employ what is called in physiology,’ diaphragm breathing which is the sinking in and bulging out of the abdomen in succession. Meanwhile the chest is kept at rest. Diaphragm breathing is employed when the body is at rest and the mind is not to emotional. In the sitting or reclining posture meditate on the rising and falling of the abdomen only, and not oh the passage of air through the nostrils. This kind of diaphragm breathing in itself prevents strong emotions from arising and is a physiological key to the prevention of the various defilements (Kilesa) from entering the mind.)

\item {} 
\sphinxAtStartPar
Instruct them to meditate on various feelings (Vedana) and acknowledge them accordingly. For example, when one is in pain, acknowledge the pain, “Painful, painful,” etc.

\item {} 
\sphinxAtStartPar
Instruct the students to meditate on thought when various ideas arise. For example, when one is thinking, acknowledge the thought (citta) “Thinking, thinking.”

\item {} 
\sphinxAtStartPar
Instruct them to meditate on the six doors of the senses, the eyes, ears, nose, tongue, body and mind, and acknowledge the perceptions thus:
\begin{enumerate}
\sphinxsetlistlabels{\arabic}{enumii}{enumiii}{}{.}%
\item {} 
\sphinxAtStartPar
While seeing, acknowledge the sight, “Seeing.”

\item {} 
\sphinxAtStartPar
While hearing acknowledge the sound, “Hearing.”

\item {} 
\sphinxAtStartPar
While smelling, acknowledge the smell, “Smelling.”

\item {} 
\sphinxAtStartPar
While tasting, acknowledge the taste, “Tasting.”

\item {} 
\sphinxAtStartPar
While experiencing a cold, hot, soft or hard touch, acknowledge the touch, “Touching.”

\item {} 
\sphinxAtStartPar
While thinking, acknowledge the thought, “Thinking.” (or imagining)

\end{enumerate}

\item {} 
\sphinxAtStartPar
Instruct the student of Insight Meditation (Vipassana) to meditate on the movements of the body and acknowledge them as described; for example; to step forward, to step backward, to turn right, to turn left, to crouch, to stretch, to hold the begging bowl, to dress, to cover the body with a blanket, to eat, to think, to chew, to taste, to discharge excretion and urine, to walk, to stand, to sit, to lie, to sleep, to wake up, to speak and to keep quiet.

\end{enumerate}

\begin{sphinxadmonition}{note}{Note:}
\sphinxAtStartPar
On the first day the instructor should examine those who are beginning the practice. If they know the Doctrine only a trifle or are old people, then he should instruct them to walk mindfully, to acknowledge the rise and the fall of the abdomen and to acknowledge various feelings and thoughts. This is enough. Subsequently the instructor can give them more instruction after again examining the state of their perceptions and mental states (or making psycho\sphinxhyphen{}analysis). This procedure applies also to young people and chidren.
\end{sphinxadmonition}

\sphinxAtStartPar
The practitioners should then prostrate themselves before their instructor in salutation and retire to their cells to begin the practice.

\sphinxAtStartPar
The instructor must go and see the students to examine their perceptions and mental states every day and give further instruction in practice according to the stage of knowledge or awareness achieved. For example when the practitioners have achieved the knowledge of discriminating mental and physical states (nāmarūpaparicchedañāna), the instructor should give further exercises, i.e. teach them to acknowledge their thoughts from then on whenever they want to crouch, to stretch or to rise.

\sphinxAtStartPar
When the practitioners have achieved the knowledge of discriminating cause and effect (paccayapariggahañana), acknowledgement of the movements while walking is increased two steps, to include, “Lifting, Treading”. In the sitting posture the practitioners should now acknowledge both the rising and falling of the belly and the posture. The important point is not to increase the number of exercises by more than two in the same day.

\sphinxAtStartPar
\sphinxstylestrong{Question:} What should the practitioners be taught?

\sphinxAtStartPar
\sphinxstylestrong{Answer:} They should perform various exercises as follows:

\sphinxstepscope


\chapter{How to Practise Meditation}
\label{\detokenize{practice:how-to-practise-meditation}}\label{\detokenize{practice::doc}}

\section{Exercise 1}
\label{\detokenize{practice:exercise-1}}\begin{enumerate}
\sphinxsetlistlabels{\arabic}{enumi}{enumii}{}{.}%
\item {} 
\sphinxAtStartPar
While sitting, meditate on the abdomen which rises on inhaling and falls on exhaling. Acknowledge the rising and falling in your mind: “Rising, Falling,” according to whether it is a rise or a fall.

\item {} 
\sphinxAtStartPar
While reclining, do the same and acknowledge in a similar manner.

\item {} 
\sphinxAtStartPar
While standing, acknowledge the posture “Standing standing.”

\item {} 
\sphinxAtStartPar
While performing the mindful walking, acknowledge in stages as follows:

\sphinxAtStartPar
When the right foot advances, acknowledge the movement, “Right goes thus,” keeping the eyes fixed on the tip of the right foot; when the left foot advances, acknowledge the movement, “Left goes thus”, keeping the eyes fixed on the tip of the left foot. Acknowledge every step in this way. Having traversed the space allowed for the mindful walk and wishing to turn back, stand still first, acknowledge the posture, “standing, standing’” then turn back slowly and composedly, and acknowledge the movement “turning, turning.” Having turned right round, stand still first, acknowledging “standing, standing”, then continue to walk mindfully, acknowledging movements as before.

\end{enumerate}

\sphinxAtStartPar
Practise each exercise until you are well experienced in it and can achieve good concentration, then pass on to the next one.


\section{Exercise 2}
\label{\detokenize{practice:exercise-2}}\begin{enumerate}
\sphinxsetlistlabels{\arabic}{enumi}{enumii}{}{.}%
\item {} 
\sphinxAtStartPar
While sitting, acknowledge the awareness in three stages, viz:\sphinxhyphen{} “Rising : falling : sitting.”

\item {} 
\sphinxAtStartPar
While reclining, also acknowledge awareness in similar stages. (See Note below).

\item {} 
\sphinxAtStartPar
While standing, acknowledge your posture, “standing, standing”, until you walk or sit down.

\item {} 
\sphinxAtStartPar
While walking mindfully, perform Exercise 1 for about 10–30 minutes. Then change your acknowledgement of the movement, i.e. when you advance your right or left foot, acknowledge the movement in two stages, “Lifting, Treading”, for about 10–30 minutes.

\end{enumerate}

\sphinxAtStartPar
\sphinxstylestrong{Thus:}
\begin{enumerate}
\sphinxsetlistlabels{\alph}{enumi}{enumii}{}{.}%
\item {} 
\sphinxAtStartPar
Acknowledge your movements while performing the mindful walk, “Right goes thus, left goes thus,” for about 10–30 minutes.

\item {} 
\sphinxAtStartPar
Acknowledge your movements while performing the mindful walk, “Lifting, Treading,” for about 10–30 minutes.

\end{enumerate}

\begin{sphinxadmonition}{note}{Note:}
\sphinxAtStartPar
(For this 2nd exercise concentrate, if only momentarily, on the postures until the images of sitting and reclining appear distinctly in your mind as if reflected in a mirror)
\end{sphinxadmonition}

\begin{sphinxadmonition}{note}{Note:}
\sphinxAtStartPar
Acknowledgement of “sitting” occurs after awareness of “falling” e.g. “Rising…… : falling…… : sitting.”

\sphinxAtStartPar
Sometimes, the abdomenal movements are not clearly recognized; when this happens it is useful to be mindful of “sitting” as soon as the awareness of “falling” has ceased or has become indistinct. The length of “falling” often varies; some examples may make the problem of acknowledgement clearer: e.g.
\begin{enumerate}
\sphinxsetlistlabels{\arabic}{enumi}{enumii}{(}{)}%
\item {} 
\sphinxAtStartPar
“Rising……… : falling……… : sitting……… : rising……… : falling……… : sitting………” and so on.

\item {} 
\sphinxAtStartPar
“Rising……… : rising…. : falling… falling… : sitting… : rising…”

\item {} 
\sphinxAtStartPar
“Rising………: rising……. : falling… falling… : falling… sitting… sitting… : rising…”

\end{enumerate}
\end{sphinxadmonition}

\sphinxAtStartPar
\sphinxstylestrong{Explanation:} Each word above (e.g. “rising”, “falling”) represents one’s acknowledgement; that is mentally saying the word. The dots(……) represent one’s awareness or mindfulness of the particular phenomenon. When awareness of one stage (such as rising) ceases (i.e. “Rising…”) acknowledgement and awareness of the next stage begins (e.g. “sitting…”).


\section{Exercise 3}
\label{\detokenize{practice:exercise-3}}\begin{enumerate}
\sphinxsetlistlabels{\arabic}{enumi}{enumii}{}{.}%
\item {} 
\sphinxAtStartPar
While sitting, acknowledge awareness in four stages, “Rising : falling : sitting : touching.” Meditate upon the touching point as a circle about the size of tical coin ( a penny). Fix your mind \sphinxhyphen{}on that point while acknowledging.

\item {} 
\sphinxAtStartPar
While reclining, give acknowledgement in four stages, “Rising : falling : reclining : touching.”

\item {} 
\sphinxAtStartPar
While standing, acknowledge your posture, “standing, standing”.

\item {} 
\sphinxAtStartPar
While performing the mindful walking, do Exercise 1 and 2 first for about 10–20 minutes each; then change the acknowledgement, and while advancing your right or left foot, acknowledge the movement in three stages, viz, “Lifting, moving, treading.”

\end{enumerate}

\sphinxAtStartPar
That is :
\begin{enumerate}
\sphinxsetlistlabels{\alph}{enumi}{enumii}{}{.}%
\item {} 
\sphinxAtStartPar
Acknowledge the movement of your feet, “Right goes thus, left goes thus,” for about 10–20 minutes.

\item {} 
\sphinxAtStartPar
Acknowledge “Lifting, treading,” for about 10–20 minutes.

\item {} 
\sphinxAtStartPar
Acknowledge “Lifting, moving, treading,” for about 10–20 minutes.

\end{enumerate}


\section{Exercise 4}
\label{\detokenize{practice:exercise-4}}\begin{enumerate}
\sphinxsetlistlabels{\arabic}{enumi}{enumii}{}{.}%
\item {} 
\sphinxAtStartPar
While sitting, acknowledge the rising and falling of the abdomen in four stages : “Rising, falling, sitting, touching”, as in Exercise 3, but now acknowledge, “Touching” several times (until the end of the out\sphinxhyphen{}going breath); i.e. “Rising, falling, touching, touching, etc.”

\item {} 
\sphinxAtStartPar
While reclining, acknowledge awareness in four stages, “Rising, falling, reclining, touching, touching, etc.“

\item {} 
\sphinxAtStartPar
While standing acknowledge your posture, “Standing standing.”

\item {} 
\sphinxAtStartPar
While performing the mindful walking, do as in Exercise 1,2, and 3 for about 10–20 minutes each, and then change the acknowledgement, i.e. while advancing with your right or left foot, acknowledge the movement in four stages, “Heel up : lifting : moving : treading,” for about 10–20 minutes

\end{enumerate}

\sphinxAtStartPar
That is:
\begin{enumerate}
\sphinxsetlistlabels{\alph}{enumi}{enumii}{}{.}%
\item {} 
\sphinxAtStartPar
Acknowledge the movement of your feet, “Right goes thus, left goes thus” for about 10–20 minutes.

\item {} 
\sphinxAtStartPar
Acknowledge, “Lifting treading,” for about 10–20 minutes.

\item {} 
\sphinxAtStartPar
Acknowledge, “Lifting, moving, treading,” for about 10–20 minutes.

\item {} 
\sphinxAtStartPar
Acknowledge, “Heel up, lifting, moving, treading,” for about 10–20 minutes.

\end{enumerate}


\section{Exercise 5}
\label{\detokenize{practice:exercise-5}}\begin{enumerate}
\sphinxsetlistlabels{\arabic}{enumi}{enumii}{}{.}%
\item {} 
\sphinxAtStartPar
While sitting, be mindful of four stages, “Rising, falling, sitting, touching”. Find the point where the touch is most distinct, and concentrate on it when acknowledging : “touching.”

\sphinxAtStartPar
For Example :
\begin{enumerate}
\sphinxsetlistlabels{\alph}{enumii}{enumiii}{}{.}%
\item {} 
\sphinxAtStartPar
“Rising, falling, sitting, touching” that is, touching with the right buttock. Concentrate on this position and acknowledge : “Touching.”

\item {} 
\sphinxAtStartPar
“Rising, falling, sitting, touching, ”that is, touching with the left buttock.

\item {} 
\sphinxAtStartPar
“Rising, falling, sitting, touching, ”that is, touching with the right knee. Concentrate on this position and acknowledge “Touching”

\item {} 
\sphinxAtStartPar
“Rising, falling, sitting, touching,” that is, touching with the left knee.

\item {} 
\sphinxAtStartPar
“Rising, falling, sitting, touching,” that is, touching with the right ankle.

\item {} 
\sphinxAtStartPar
“Rising, falling, sitting, touching,” that is, touching with the left ankle.

\end{enumerate}

\item {} 
\sphinxAtStartPar
While reclining, acknowledge in four stages, viz. “Rising, falling, reclining, touching”.

\item {} 
\sphinxAtStartPar
While standing, acknowledge your posture, “Standing, standing,”

\item {} 
\sphinxAtStartPar
While performing the mindful walking, do as in Exercises 1,2,3 and 4 for about 10–20 minutes each, and then change the acknowledgement i.e. While advancing the right or left foot acknowledge the movements in five stages heel up, lifting, moving, dropping, treading“, for about 10–20 minutes.

\end{enumerate}

\sphinxAtStartPar
To Summarize :
\begin{enumerate}
\sphinxsetlistlabels{\alph}{enumi}{enumii}{}{.}%
\item {} 
\sphinxAtStartPar
Acknowledge your movements in the mindful walking, “Right goes thus, left goes thus,” For about 10–20 minutes.

\item {} 
\sphinxAtStartPar
Acknowledge, “Lifting, Treading,” for about 10–20 minutes.

\item {} 
\sphinxAtStartPar
Acknowledge, “Lifting, moving, treading,” for about 10–20 minutes.

\item {} 
\sphinxAtStartPar
Acknowledge, “Heel up, lifting, moving, treading,” for about 10–20 minutes.

\item {} 
\sphinxAtStartPar
Acknowledge, “Heel up, lifting, moving, dropping treading,” for about 10–20 minutes.

\end{enumerate}


\section{Exercise 6}
\label{\detokenize{practice:exercise-6}}\begin{enumerate}
\sphinxsetlistlabels{\arabic}{enumi}{enumii}{}{.}%
\item {} 
\sphinxAtStartPar
While sitting be mindful as follows :
\begin{enumerate}
\sphinxsetlistlabels{\alph}{enumii}{enumiii}{}{.}%
\item {} 
\sphinxAtStartPar
“Rising, falling, sitting, touching,” that is, touching with the right buttock.

\item {} 
\sphinxAtStartPar
“Rising, falling, sitting, touching,” that is, touching with the left buttock.

\item {} 
\sphinxAtStartPar
“Rising, falling, sitting, touching,” that is, touching with the right knee.

\item {} 
\sphinxAtStartPar
“Rising, falling, sitting, touching,” that is, touching with the left knee.

\item {} 
\sphinxAtStartPar
“Rising, falling, sitting, touching,” that is, touching with the right ankle,

\item {} 
\sphinxAtStartPar
“Rising, falling, sitting, touching,” that is, touching with the left ankle.

\item {} 
\sphinxAtStartPar
“Rising, falling, sitting, touching,” that is, touching at various points along the body.

\end{enumerate}

\item {} 
\sphinxAtStartPar
While reclining, acknowledge thus : “Rising, falling, reclining, touching,” etc.

\item {} 
\sphinxAtStartPar
While standing, acknowledge your posture, “Standing, standing”.

\item {} 
\sphinxAtStartPar
While performing the mindful walking, acknowledge the movements :
\begin{enumerate}
\sphinxsetlistlabels{\alph}{enumii}{enumiii}{}{.}%
\item {} 
\sphinxAtStartPar
“Right goes thus, left goes thus,” for about 5–10 minutes.

\item {} 
\sphinxAtStartPar
“Lifing, treading,” for about 5–10 minutes.

\item {} 
\sphinxAtStartPar
“Lifing, moving, treading,” for about 5–10 minutes.

\item {} 
\sphinxAtStartPar
“Heel up lifting, moving, treading,” for about 5–10 minutes.

\item {} 
\sphinxAtStartPar
“Heel up lifting, moving, dropping treading,” for about 5–10 minutes.

\item {} 
\sphinxAtStartPar
Now acknowledge a further stage : “Heel up : lifting : moving : dropping : touching : pressing.” for about 10–20 minutes.

\end{enumerate}

\end{enumerate}


\section{Exercise 7}
\label{\detokenize{practice:exercise-7}}\begin{enumerate}
\sphinxsetlistlabels{\arabic}{enumi}{enumii}{}{.}%
\item {} 
\sphinxAtStartPar
Having performed the mindful walking to the extremity of the space allowed, stop to turn back, Before stopping, however, acknowledge your wish, “Wishing to stop,” and having stopped, acknowledge the action, “Stopped, stopped.” Before turning back. acknowledge your desire, “Wishing to turn, wishing to turn” and during turning round, acknowledge your action in steps “Turning, turning,”. Then stand still and acknowledge your posture, “Standing, standing”. Next perform the mindful walking again and acknowledge the movements as before.

\item {} 
\sphinxAtStartPar
When a desire arises to look right or left, acknowledge it thus: “Wishing to look aside, wishing to look aside”. Wishing to look aside, acknowledge the movement, “looking aside, looking aside.”.

\item {} 
\sphinxAtStartPar
Before bending or stretching, acknowledge your wish, “Wishing to bend,”. Or “Wishing to stretch,”. While actually doing the action, acknowledge it, “Bending, bending,” or “Stretching, streching,”.

\item {} 
\sphinxAtStartPar
Before grasping anything such as clothes, blankets, begging bowls, pots, jugs, and plates, acknowledge your wish, “Seeing, wishing to grasp.” While moving your hand, acknowledge the action, “Moving, moving,” While touching with your hand, acknowledge the action, “Touching.” While grasping it and moving it towards you, acknowledge the action, “Bringing, bringing”.

\item {} 
\sphinxAtStartPar
While you are eating or drinking or chewing or tasting or licking, acknowledge the action in similar manner.

\sphinxAtStartPar
For Example:
\begin{enumerate}
\sphinxsetlistlabels{\alph}{enumii}{enumiii}{}{.}%
\item {} 
\sphinxAtStartPar
While perceiving the food, acknowledge the action. “Perceiving, Preceiving.”

\item {} 
\sphinxAtStartPar
While desiring to eat it, acknowledge the wish, “Desiring, Desiring.”

\item {} 
\sphinxAtStartPar
While advancing your hand towards it, ackowledge the action, “Moving, moving.”

\item {} 
\sphinxAtStartPar
While touching it, acknowledge the action “Touching, touching.”

\item {} 
\sphinxAtStartPar
While grasping or holding it, acknowledge the action, “Grasping” or “holding,”

\item {} 
\sphinxAtStartPar
While lifting it, acknowledge the action, “Lifting.”

\item {} 
\sphinxAtStartPar
While opening your mouth, acknowledge the action, “Opening.”

\item {} 
\sphinxAtStartPar
While the food is touching your mouth, acknowledge “Touching.”

\item {} 
\sphinxAtStartPar
While chewing, acknowledge the action, “Chewing.”

\item {} 
\sphinxAtStartPar
While swallowing, acknowledge the action, “Swallowing.”

\item {} 
\sphinxAtStartPar
While completing the eating, acknowledge the action, “Completing.”

\end{enumerate}

\item {} 
\sphinxAtStartPar
While wishing to discharge excrement or urine, acknowledge your thought, “Wishing to excrete.” While excreting, acknowledge the action, “Excreting.”

\item {} 
\sphinxAtStartPar
When wishing to walk, stand, sit, sleep, get up, speak or keep silent, acknowledge the thoughts, “Wishing to walk,” “Wishing to stand,” “Wishing to sit,” “Wishing to sleep,” “Wishing to get up,” “Wishing to speak,” or “Wishing to keep silent.”

\end{enumerate}


\section{Exercise 8}
\label{\detokenize{practice:exercise-8}}\begin{enumerate}
\sphinxsetlistlabels{\arabic}{enumi}{enumii}{}{.}%
\item {} 
\sphinxAtStartPar
When seeing, acknowledge the perception, “Seeing, seeing.”

\item {} 
\sphinxAtStartPar
When hearing, acknowledge the perception, “Hearing, hearing.”

\item {} 
\sphinxAtStartPar
When smelling, acknowledge “Smelling, smelling.”

\item {} 
\sphinxAtStartPar
When tasting, acknowledge “Tasing, tasing.”

\item {} 
\sphinxAtStartPar
When touching, acknowledge “Touching, touching.”

\item {} 
\sphinxAtStartPar
When thinking, acknowledge either “Thinking, thinking,” or “Imagining, imagining.”

\end{enumerate}


\section{Exercise 9}
\label{\detokenize{practice:exercise-9}}\begin{enumerate}
\sphinxsetlistlabels{\arabic}{enumi}{enumii}{}{.}%
\item {} 
\sphinxAtStartPar
While acknowledging the rising and falling of the adbomen in the sitting posture, “Rising, Falling.” if any pain occurs, stop for a while, and acknowledge the pain, ache or stiffness, “Painful,” “aching” or “stiffness”. If the pain is too great to bear, stop the acknowledgement and go back to acknowledging the rising and falling of the abdomen If the pain is still there, change your posture.

\item {} 
\sphinxAtStartPar
If comfort arises, acknowledge it, “Comfort arising”

\item {} 
\sphinxAtStartPar
While reclining or standing, if any comfort or discomfort or indifference arises, acknowledge it, “Comfort arising” or “discomfort arising” or “indifference arising.”

\end{enumerate}

\sphinxAtStartPar
If any pain arises during the mindful walk, stop first; then acknowledge the pain as described before,
Note : If any mental image (Nimitta) such as light or a mountain arises, acknowledge it, “Seeing, seeing.” until it vanishes.


\section{Exercise 10}
\label{\detokenize{practice:exercise-10}}\begin{enumerate}
\sphinxsetlistlabels{\arabic}{enumi}{enumii}{}{.}%
\item {} 
\sphinxAtStartPar
While sitting, if a need for something arises’ acknowledge it, “Needing, needing” or, “Desiring, desiring.”

\item {} 
\sphinxAtStartPar
If you wish to leave practice through, for example, boredom, or if you see or think of something and feel aversion, acknowledge your thought, e.g. “Discontented,” or “Hating.”

\item {} 
\sphinxAtStartPar
If you fell sleepy, acknowledge your feeling, “Sleepy”

\item {} 
\sphinxAtStartPar
If your mind is distracted, acknowledge your feeling, “Distracted.”

\item {} 
\sphinxAtStartPar
If you have any doubt, acknowledge your thought, “Doubting.”

\item {} 
\sphinxAtStartPar
If greed, anger, distraction and doubt, as examples of mental conditions, clear away, acknowledge that also.

\item {} 
\sphinxAtStartPar
While performing the mindful walking, if the mind is distracted stop walking and acknowledge your thought, “Distracted.” After the distraction has cleared away, go on with the mindful\sphinxhyphen{}walking.

\end{enumerate}


\section{Exercise 11}
\label{\detokenize{practice:exercise-11}}\begin{enumerate}
\sphinxsetlistlabels{\arabic}{enumi}{enumii}{}{.}%
\item {} 
\sphinxAtStartPar
If the mind is contented in sight, sound, smell, taste, touch, try to realize that it is a sensual contentment (Kāmagunā). Acknowledge your feeling, “Contented.”

\item {} 
\sphinxAtStartPar
When aversion arises, try to realize that it is hatred or a wish for revenge. Acknowledge it “Hating” or “Revenge.”

\item {} 
\sphinxAtStartPar
When the mind is jaded or apathetic, try to realize that this feeling is torpor and languor (Thinamidha). Acknowledge it, e.g. “Sleepy”.

\item {} 
\sphinxAtStartPar
If the mind is distracted, worried or depressed, try to realise that distraction and worry (Uddhaccakukkucca) have arisen, and acknowledge such feelings. “Distracted”, or Worrying“, or ”Depressed“.

\item {} 
\sphinxAtStartPar
When doubts in respect of mental and physical states (nāmarūpa), ultimate reality and the concepts (paññātti) arise, try to realise that this is sceptical doubt (Vicikicchā). Acknowledge it, “Doubting.”

\end{enumerate}


\section{Exercise 12}
\label{\detokenize{practice:exercise-12}}\begin{enumerate}
\sphinxsetlistlabels{\arabic}{enumi}{enumii}{}{.}%
\item {} 
\sphinxAtStartPar
Before sitting down, acknowledge your thought, “Wishing to sit down.” Then lower yourself slowly in stages and acknowledge the action, “Sitting down” until you touch the floor. Do the acknowledgement in 8–9–10 steps.

\item {} 
\sphinxAtStartPar
While acknowledging “Rising, falling, sitting, touching,” and an itch arises, acknowledge it, “Itching.” After the acknowledgement if the itch is still there and you want to scratch, acknowledge your desire “Wishing to scratch.” When your hand touches the spot, acknowledge the action, “Scratching.” When the itch disappears, acknowledge it, “disappearing,” and when you lower your hand from the spot, acknowledge your action, “Lowering,” until it is where it used to be. Then begin to concentrate on the rise and fall of the abdomen again and acknowledge your awareness, “Rising, falling, sitting, touching,”

\end{enumerate}


\section{Exercise 13}
\label{\detokenize{practice:exercise-13}}\begin{enumerate}
\sphinxsetlistlabels{\arabic}{enumi}{enumii}{}{.}%
\item {} 
\sphinxAtStartPar
Before beginning the meditation, make a wish as follows:

\sphinxAtStartPar
“May I be clearly aware of the coming\sphinxhyphen{}into\sphinxhyphen{}being and passing\sphinxhyphen{}away of all mental and physical phenomena appearing to the mind during twenty\sphinxhyphen{}four hours.”

\end{enumerate}

\sphinxAtStartPar
Make this wish this whenever you wish, but spend at least twenty\sphinxhyphen{}four hours in meditation during this exercise.
\begin{enumerate}
\sphinxsetlistlabels{\arabic}{enumi}{enumii}{}{.}%
\setcounter{enumi}{1}
\item {} 
\sphinxAtStartPar
Having made the wish as above, perform the mindful walking first; then sit down and acknowledge the rising and falling of the abdomen, “Rising, falling, sitting, touching,” as described before. Perform the two exercise in alternation throughout the twenty\sphinxhyphen{}four hours.

\end{enumerate}


\section{Exercise 14}
\label{\detokenize{practice:exercise-14}}\begin{enumerate}
\sphinxsetlistlabels{\arabic}{enumi}{enumii}{}{.}%
\item {} 
\sphinxAtStartPar
Perform the mindful walking first, then proceed as follows:
\begin{enumerate}
\sphinxsetlistlabels{\alph}{enumii}{enumiii}{}{.}%
\item {} 
\sphinxAtStartPar
Make a wish that in a period of one hour, the phenomena of arising and ceasing shall appear at least five times.

\item {} 
\sphinxAtStartPar
If within this hour the phenomena of arising and ceasing appear distinctly, at least five times and possibly as many as sixty\sphinxhyphen{}five times reduce the period of the exercise to 30 minutes and make the wish that within these thirty minute the phenomena of arising and ceasing shall appear to you several times.

\item {} 
\sphinxAtStartPar
Make a wish in the same manner and reduce the period of the exercise down to 20–15–10–5 minutes. Within 5 minutes the phenomenon should appear at least twice, but it may appear as many as six times.

\end{enumerate}

\item {} 
\sphinxAtStartPar
Alternate Walking and sitting exercises for twenty\sphinxhyphen{}four hours.

\end{enumerate}


\section{Exercise 15}
\label{\detokenize{practice:exercise-15}}\begin{enumerate}
\sphinxsetlistlabels{\arabic}{enumi}{enumii}{}{.}%
\item {} 
\sphinxAtStartPar
Perform the mindful walking first ; then in the sitting posture make a wish to attain steady concentration for 5 minutes. Next acknowledge “Rising, falling, sitting, touching,” etc. The resolution is fulfilled if the mind abides in concentration and becomes unconscious of outside phenomena for a 5 full minutes. Keep a careful check on the time and if this exercise cannot be continued for 5 minutes, repeat it until you are successful. Then try to increase the period of full concentration.

\item {} 
\sphinxAtStartPar
Make a resolve to obtain steady concentration without consciousness of outside phenomena for 10 muinutes. If this cannot be achieved yet, try again until you are quite experienced. Then practise further for 15–20–30 minutes to 1 hour, one and a half hour, 2–3–4–5–6–7–8 hours, up to 24 hours.

\item {} 
\sphinxAtStartPar
The number of minutes and hours is to be reckoned from the point of steady concentration with nonconsciousness onwards. In such a condition, we do not experience any feeling. The period wished for being fulfilled, consciousness will return of its own accord as in waking, but this is not waking.

\end{enumerate}


\section{Exercise 16}
\label{\detokenize{practice:exercise-16}}
\sphinxAtStartPar
Practitioners who have become experienced in practice and would like to qualify as future instructors should perform a Special exercise as follows:

\sphinxAtStartPar
\sphinxstylestrong{First exercise,} to be done in one day.
\begin{enumerate}
\sphinxsetlistlabels{\arabic}{enumi}{enumii}{}{.}%
\item {} 
\sphinxAtStartPar
Perform the mindful walking first, then sit as usual and resolve that within one hour the mental and physical states in the process of arising and ceasing shall appear distinctly. Acknowledge ilie awareness, “Rising, falling, sitting, touching”, etc. to complete one full hour. While acknowledging, you will perceive the arising ,md ceasing of the mental and physical states more distinctly than before. This insight knowledge is called Udyabbayañāṇa.

\item {} 
\sphinxAtStartPar
After this, resolve that within the succeeding hour only mental and physical states in cessation (or in their passing\sphinxhyphen{}away) shall be perceived. Then acknowledge the awareness, “Rising, lulling, sitting, touching,” etc. to complete one full hour. While acknowledging, only the passing\sphinxhyphen{}away of the mental and physical states will appear, that is, the cessation appears more distinctly than before. This insight is called Bhaṃganñāṇa.

\end{enumerate}

\sphinxAtStartPar
\sphinxstylestrong{Second exercise,} to be done in one day.
\begin{enumerate}
\sphinxsetlistlabels{\arabic}{enumi}{enumii}{}{.}%
\item {} 
\sphinxAtStartPar
Perform the mindful walking first, then sit as usual and resolve that within one hour the Bhayañāna shall arise in you. Acknowledge the perceptions, “Rising, falling, sitting, touching,” etc. to complete one full hour. While thus acknowledging, fear will arise in your mind. This insight knowledge is therefore known as Bhayañāṇa.

\item {} 
\sphinxAtStartPar
In the succeeding hours, resolve that the Adinavanana shall arise, and acknowledge the perceptions, “Rising, falling, sitting, touching,” etc. to complete one full hour. While acknowledging in the sitting posture, there will arise all kinds of afflictions latent in the mental and physical states, such as pain, aching, impermanence, suffering, and anattā. This knowledge is known Adinavañāṇa.

\item {} 
\sphinxAtStartPar
In the third successive hour resolve that the Nibbid āñāna shall arise. Acknowledge the awareness, “Rising, falling, sitting, touching,” to complete one full full hour. While acknowledging in the sitting posture there will arise revulsion, the mental and physical states appear to you as ugly refuse, full of afflictions and suffering, unpleasant and disgusting, This knowledge is called Nibbidāñāṇa.

\end{enumerate}

\sphinxAtStartPar
\sphinxstylestrong{Third exercise,} to be done in one day.
\begin{enumerate}
\sphinxsetlistlabels{\arabic}{enumi}{enumii}{}{.}%
\item {} 
\sphinxAtStartPar
Perform the mindful walking first ; then sit as usual and resolve that within this hour the Muñcitukamyatāñana shall arise. Acknowledge the perceptions “Rising, falling, sitting, touching”, to complete one full hour. While acknowledging in the sitting posture, there will arise a wish to retire, to escape into seclusion. This knowledge is known as Muñcitukamyatāñāṇa.

\item {} 
\sphinxAtStartPar
In the succeeding hour resolve that within this hour the Patisamkhañāna shall arise and acknowledge the perceptions, “Rising, falling, sitting, touching,” etc. to complete one full hour. While acknowledging in the sitting posture, there will arise an effort to use one’s energy to seek detachment and to escape into seclusion. This knowledge is Patisamkhanñāṇa.

\item {} 
\sphinxAtStartPar
In the third successive hour resolve that within this hour the Sarnkharūpekkhañāna, shall arise and acknowledge the perceptions. “Rising, falling, sitting, touching,” etc., to complete one full hour. While acknowledging in this posture there will arise equanimity with regard to mental and physical states. This knowledge is know as Sarnkharūpekkhañāṇa.

\end{enumerate}


\bigskip\hrule\bigskip


\sphinxAtStartPar
\sphinxstylestrong{Q:} What are the benefits of performing Insight Meditation (Vipassanā) in the way described? Please expound a little further.

\sphinxAtStartPar
\sphinxstylestrong{A:} There are several benefits as follows :
\begin{enumerate}
\sphinxsetlistlabels{\arabic}{enumi}{enumii}{}{.}%
\item {} 
\sphinxAtStartPar
To give certainty of Truth, and not to be deceived by and not to hold fast to concepts (paññāat ti) which are mere mundane conventions.

\item {} 
\sphinxAtStartPar
To make people truly cultured, having good morals.

\item {} 
\sphinxAtStartPar
To make people love one another, make them feel their unity and to be compassionate towards each other, and to make them have gladness and appreciation when they see others who are joyful.

\item {} 
\sphinxAtStartPar
To bring about a better standard of human behaviour.

\item {} 
\sphinxAtStartPar
To make people know themselves and how to govern themselves.

\item {} 
\sphinxAtStartPar
To cultivate humility.

\item {} 
\sphinxAtStartPar
To bring about realisation of human unity.

\item {} 
\sphinxAtStartPar
To make people abide in gratitude.

\item {} 
\sphinxAtStartPar
To make people Bhikkhus of the Ariya Sangha, as this practice of Dhamma is for the following attainments :\sphinxhyphen{}
\begin{enumerate}
\sphinxsetlistlabels{\alph}{enumii}{enumiii}{}{.}%
\item {} 
\sphinxAtStartPar
To be without the five hindrances (Nivarana)

\item {} 
\sphinxAtStartPar
To be without the five stands of sensual pleasure (Kamaguna)

\item {} 
\sphinxAtStartPar
To be without the factors of the “fivefold clinging to existence” (Upādānakakhandha)

\item {} 
\sphinxAtStartPar
To be without the five lower fetters ; the Ego\sphinxhyphen{}illusion (Sakkayaditthi), sceptical doubt (Vicikicchā), attachment (or clinging) to mere rules and ritual (Silabbataparamasa), sensuous desire (Kamachanda) and ill\sphinxhyphen{}will (Vyāpada)

\item {} 
\sphinxAtStartPar
To be free from the 5\sphinxhyphen{}fold destinies (gati).

\item {} 
\sphinxAtStartPar
To be without selfishness in any form ; selfishness in lodgings, selfishness in family, selfishness in property, selfishness in rank and selfishness in Dhamma.

\item {} 
\sphinxAtStartPar
To be without the five higher fetters, set out as craving for life in the world of pure form (Rūparāga), craving for the formless world (Arūparāga), pride, distraction and ignorance.

\item {} 
\sphinxAtStartPar
To be without Cetokhila, the five “nails” which limit the mind, comprising doubt in the Buddha the Dhamma and the Sangha, and in the training and anger against one’s fellow\sphinxhyphen{}monks.

\item {} 
\sphinxAtStartPar
To be without Cetovinibandha, the five fetters which hinder the mind from making right exertion ; namely : lust for sensuous objects, for the body, for visible things, for eating and sleeping, and leading the monk’s life for the sake of heavenly rebirth.

\item {} 
\sphinxAtStartPar
To be free from sorrow, grief, woe and lamentation and attain the Path, Fruition and Nibbāna.

\item {} 
\sphinxAtStartPar
The highest blessing is to succeed as an adept or Arahant. Of lower qualifications are the Never Returner (Anagami), the Once Returner (Sakadagami), and the Stream Winner (Sotapana). Still lower down on the scale are the commoners who have a steady determination to go along the path of righteousness according to the principle:
\begin{quote}

\sphinxAtStartPar
\sphinxstyleemphasis{“Iminā pana ñāñanena samannāgato vipassako Buddhasāsane laddhassāsa laddhapatittho niyagatiko jūlasotapanno nama hoti,”} meaning, “When the practitioners who are endowed with wisdom have practised Insight Meditation (Vipassanā), they will succeed as minor Sotāpanas, become light hearted, obtain true refuge in Dhamma and have a steady determination to go along the path of righteousness.”
\end{quote}

\sphinxAtStartPar
Also, the practitioners who have practised Insight Medetation (Vipassana) and gained insight into the arising and ceasing of mental and physical states, are to be considered as blessed as stated thus :

\sphinxAtStartPar
\sphinxstyleemphasis{“Yoca vassasatam jive apassam udyabbayam akāham jivitam seyyo passto udyabbayam”} “Those who perceive the arising and ceasing of mental and physical states, even though they live for a day only, are far better than those who never see the arising and ceasing of mental and physical states and live a hundred years.“

\end{enumerate}

\end{enumerate}

\sphinxAtStartPar
\sphinxstylestrong{Q:} How long would it take to succeed in the practice of Insight meditation?

\sphinxAtStartPar
\sphinxstylestrong{A:} If the practice is done continuously for 1 day 15 days 1 month, 2–3–4–5–6–7 months or 1–2–3–4–5–6–7 years one would succeed according to whether one’s previous merits are strong or weak. The time specified is for the practitioners of medium previous merits. Those with great previous merits, when they practise in the morning, can succeed in the evening, and when they practise in the evening, can succeed in the morning according to the words of the commentator:
\begin{quote}

\sphinxAtStartPar
\sphinxstyleemphasis{“Tikkhapaññam pana sandhāya pātova anusittho sayaṃ visesam adhigāissati   sāyam anusittho  pato   visesam adhigmissatiti vuttam”}

\sphinxAtStartPar
“Thus was it said ; to be instructed in the morning and to attain the divine Dhamma in the evening, to be instructed in the evening and to attain the divine Dhamma in the morning, is the way of Tikkha persons endowed with great previous merits”.
\end{quote}

\sphinxAtStartPar
\sphinxstyleemphasis{Explanation of the practice of insight meditation (vipassana) ends here.}


\section{Translator’s note}
\label{\detokenize{practice:translator-s-note}}
\sphinxAtStartPar
This booklet was originally produced by the Central office of the Division of Vipassana Dhura at Mahadhat Monastery several years ago. When the printing committee of Wat Mahadhat decided to reprint it I was asked to check the English. It became obvious that simple correction would not be sufficient, and rewriting was in order. I have frequently referred to the original Thai text, which is remarkably clear and well\sphinxhyphen{}organized, in order to clarify several sections of the text and fill in parts which were missing. Obviously some of the experiences described in this book defy expression in any language, but I have aimed for clarity whenever possible.

\sphinxAtStartPar
This part of the book is not intended for people with little experience of meditation. It is intended for meditation teachers and experienced non\sphinxhyphen{}Thai meditators who may find it difficult to get a good translation by a competent interpreter.

\begin{DUlineblock}{0em}
\item[] \sphinxstyleemphasis{Helen Jandamit}
\item[] \sphinxstyleemphasis{Vorasak Jandamit}
\end{DUlineblock}

\sphinxstepscope


\chapter{Manual for checking your vipassana kamatthana progress}
\label{\detokenize{progress:manual-for-checking-your-vipassana-kamatthana-progress}}\label{\detokenize{progress::doc}}

\section{Nāmarupa pariccheda Ñana}
\label{\detokenize{progress:namarupa-pariccheda-nana}}
\sphinxAtStartPar
In this Ñana or state of wisdom or knowledge, the meditator is able to distinguish NAMA from RUPA For example, he is aware that the rising and falling movements of the abdomen are RUPA and the mind which acknowledges these movements is NAMA. A movement of the foot is RUPA and the consciousness of that movement is NAMA.

\sphinxAtStartPar
The meditator can distinguish between NAMA and RUPA with regard to the five senses as follows  :\sphinxhyphen{}
\begin{enumerate}
\sphinxsetlistlabels{\arabic}{enumi}{enumii}{}{.}%
\item {} 
\sphinxAtStartPar
When seeing a form, the eyes and the colour are RUPA and the consciousness of the seeing is NAMA.

\item {} 
\sphinxAtStartPar
When hearing a sound, the sound itself and the hearing are RUPA and consciousness of the hearing is NAMA.

\item {} 
\sphinxAtStartPar
When smelling something, the smell itself and the nose are RUPA, and the consciousness of the smell is NAMA.

\item {} 
\sphinxAtStartPar
When tasting something, the taste and the tongue are RUPA, and the consciousness of taste is NAMA.

\item {} 
\sphinxAtStartPar
When touching something, whatever is cold, hot, soft or hard to the touch is RUPA and consciousness of the contact is NAMA.

\end{enumerate}

\sphinxAtStartPar
In conclusion, in this Ñana the meditator realizes that the whole body is RUPA and the mind (or consciousness of the sensations of the body) is NAMA. only NAMA and RUPA exist. There is no being, no individual self, no T, no ‘he’ or ‘she’ etc. When sitting, the body and its movement are RUPA and awareness of the sitting is NAMA. The act of standing is RUPA and awareness of the standing is NAMA. The act of walking is RUPA and the awareness of the walking is NAMA.


\section{Paccaya pariggaha Ñana}
\label{\detokenize{progress:paccaya-pariggaha-nana}}
\sphinxAtStartPar
\sphinxstyleemphasis{(Knowledge of discerning the condition of mentality/materiality)}

\sphinxAtStartPar
In some instances RUPA is the cause and NAMA is the effect, for example when the abdomen rises and consciousness follows. At other times NAMA is the cause and RUPA is the effect, for example the wish to sit is the cause and the sitting is’ the effect ; in other words volitional activity precedes physical action.

\sphinxAtStartPar
Some symptoms of this ÑANA
\begin{enumerate}
\sphinxsetlistlabels{\alph}{enumi}{enumii}{}{.}%
\item {} 
\sphinxAtStartPar
The abdomen may rise but fail to fall.

\item {} 
\sphinxAtStartPar
The abdomen may fall deeply and remain in that position.

\item {} 
\sphinxAtStartPar
The rising and falling of the abdomen seems to have disappeared but when touched by the hand movements can still be felt.

\item {} 
\sphinxAtStartPar
At times there are feelings of distress of varing intensity.

\item {} 
\sphinxAtStartPar
Some meditators may be much disturbed by visions or hallucinations.

\item {} 
\sphinxAtStartPar
The rising and falling of the abdomen and the acknowledgement of the movements function at the same time.

\item {} 
\sphinxAtStartPar
One may be startled sometimes bending forwards or backwards.

\item {} 
\sphinxAtStartPar
The meditator conceives that this existance, the next and all existances only derive from the interaction of cause and effect. They consist of NAMA and RUPA only.

\item {} 
\sphinxAtStartPar
A single rise of the abdomen has two stages.

\end{enumerate}


\section{Sammasana Ñana}
\label{\detokenize{progress:sammasana-nana}}
\sphinxAtStartPar
\sphinxstyleemphasis{(Knowledge of comprehending mentality/materiality as unsatisfactory and not\sphinxhyphen{}self)}

\sphinxAtStartPar
Some characteristics of this ÑANA  :\sphinxhyphen{}
\begin{enumerate}
\sphinxsetlistlabels{\alph}{enumi}{enumii}{}{.}%
\item {} 
\sphinxAtStartPar
One considers NAMA and RUPA through the five senses. One is aware of the three characteristics, ANICCA (impermance) DUKKHA (suffering) and ANATTA (non\sphinxhyphen{}self) which are   referred to collectively as TRILAKKHANA.

\item {} 
\sphinxAtStartPar
One rising movement of the abdomen has three sections, called UPADHA, DTITI and BHANKA or coming into existance, continuity and vanishing. One falling movement of the abdomen has three sections.

\item {} 
\sphinxAtStartPar
There are feelings of distress which disappear only slowly; after seven or eight acknowledgements.

\item {} 
\sphinxAtStartPar
There are many NIMITTAS (mental images) which disappear slowly after several acknowledgements).

\item {} 
\sphinxAtStartPar
The rising and falling movements of the abdomen may disappear for either a long or short interval.

\item {} 
\sphinxAtStartPar
Breathing may be fast, slow or obstructed.

\item {} 
\sphinxAtStartPar
The mind may be distracted which shows that it is aware of the TRILAKKHANA or the three characteristics.

\item {} 
\sphinxAtStartPar
The meditator’s hands or feet may clench or twitch.

\item {} 
\sphinxAtStartPar
Some of the ten VIPASSANUPAKILESAS (defilements of insight) may appear in this ÑANA.

\end{enumerate}


\subsection{The Ten Vipassanupakilesas}
\label{\detokenize{progress:the-ten-vipassanupakilesas}}

\subsubsection{OBHASA (Illumination)}
\label{\detokenize{progress:obhasa-illumination}}
\sphinxAtStartPar
The meditator may be aware of the following manifestations of light :\sphinxhyphen{}
\begin{enumerate}
\sphinxsetlistlabels{\alph}{enumi}{enumii}{}{.}%
\item {} 
\sphinxAtStartPar
He may be aware of light similar to that of firefly, a torch or a car headlamp.

\item {} 
\sphinxAtStartPar
The whole room may be lit up sufficiently to enable the meditator to see his body.

\item {} 
\sphinxAtStartPar
He may be aware of light which seems to pass through the wall.

\item {} 
\sphinxAtStartPar
There may be a light which enables him to see various places in front of him.

\item {} 
\sphinxAtStartPar
There may be a bright light as though a door has come open. Some meditators lift up their hands as if to shut it, others open their eyes to see what caused the light.

\item {} 
\sphinxAtStartPar
A vision of brightly coloured flowers surrounded by light may be seen.

\item {} 
\sphinxAtStartPar
Miles and miles of sea may be seen.

\item {} 
\sphinxAtStartPar
Rays of light seem to be emitted from the meditators heart and body.

\item {} 
\sphinxAtStartPar
Hallucinations, such as seeing an elephant, may occur.

\end{enumerate}


\subsubsection{PĪTI (Joy or rapture)}
\label{\detokenize{progress:piti-joy-or-rapture}}
\sphinxAtStartPar
There are five kinds of PITI : KHUDDAKA (minor rapture) ; KHANIKA (momentary joy) ; OKKANTIKA (flood of joy) ; UBBENKA (uplifting joy) ; and PHARANA (pervading rapture).
\begin{enumerate}
\sphinxsetlistlabels{\arabic}{enumi}{enumii}{}{)}%
\item {} 
\sphinxAtStartPar
KHUDDAKA PĪTI (Minor rapture)

\sphinxAtStartPar
This state is characterized by the following :\sphinxhyphen{}
\begin{enumerate}
\sphinxsetlistlabels{\alph}{enumii}{enumiii}{}{.}%
\item {} 
\sphinxAtStartPar
The meditator may be aware of white colour.

\item {} 
\sphinxAtStartPar
There may be a feeling of coolness or dizziness and the hairs of the body may stand on end.

\item {} 
\sphinxAtStartPar
Tears may fall and there may be feelings of terror.

\end{enumerate}

\item {} 
\sphinxAtStartPar
KHANIKĀ PĪTI (Momentary joy)

\sphinxAtStartPar
Characteristics of this Piti include :\sphinxhyphen{}
\begin{enumerate}
\sphinxsetlistlabels{\alph}{enumii}{enumiii}{}{.}%
\item {} 
\sphinxAtStartPar
Flashes of light light may be seen.

\item {} 
\sphinxAtStartPar
Sparks of light may be seen.

\item {} 
\sphinxAtStartPar
Nervous twitching may occur.

\item {} 
\sphinxAtStartPar
There may be a feeling of stiffness all over the body.

\item {} 
\sphinxAtStartPar
There may be a feeling as if ants are climbing over the body.

\item {} 
\sphinxAtStartPar
The meditator may feel hot all over his body.

\item {} 
\sphinxAtStartPar
The meditator may shiver.

\item {} 
\sphinxAtStartPar
Various red colours may be seen.

\item {} 
\sphinxAtStartPar
Body hair may rise slightly.

\item {} 
\sphinxAtStartPar
The meditator may feel itchy as if ants are scrambling on his face and body.

\end{enumerate}

\item {} 
\sphinxAtStartPar
OKKANTIKA PĪTI (Flood of joy)

\sphinxAtStartPar
Characteristics of this Piti include :\sphinxhyphen{}
\begin{enumerate}
\sphinxsetlistlabels{\alph}{enumii}{enumiii}{}{.}%
\item {} 
\sphinxAtStartPar
The body may shake and tremble.

\item {} 
\sphinxAtStartPar
The face, hands and feet may twitch

\item {} 
\sphinxAtStartPar
There may be violent shaking as if the bed is going to turn upside down.

\item {} 
\sphinxAtStartPar
Nausea and at times actual vomiting occurs.

\item {} 
\sphinxAtStartPar
There may be a rhythmic feeling like waves breaking on the shore.

\item {} 
\sphinxAtStartPar
Ripples of energy may seem to flow over the body.

\item {} 
\sphinxAtStartPar
The body may vibrate like a stick which is fixed in a flowing stream.

\item {} 
\sphinxAtStartPar
A light yellow colour may be observed.

\item {} 
\sphinxAtStartPar
The body may bend to and fro.

\end{enumerate}

\item {} 
\sphinxAtStartPar
UBBENKA PĪTI (Uplifting joy)

\sphinxAtStartPar
Characteristics of this Piti include  :\sphinxhyphen{}
\begin{enumerate}
\sphinxsetlistlabels{\alph}{enumii}{enumiii}{}{.}%
\item {} 
\sphinxAtStartPar
The body feels as if it is extending or moving upwards.

\item {} 
\sphinxAtStartPar
There may be a feeling as though lice are climbing on the face and body.

\item {} 
\sphinxAtStartPar
Diarrhea may occur.

\item {} 
\sphinxAtStartPar
The body may bend forwards or backwards.

\item {} 
\sphinxAtStartPar
One may feel that one’s head has been moved backwards and forwards by somebody.

\item {} 
\sphinxAtStartPar
There may be a chewing movement with the mouth either open or closed.

\item {} 
\sphinxAtStartPar
The body sways like a tree being blown by the wind,

\item {} 
\sphinxAtStartPar
The body bends forwards and may fall down.

\item {} 
\sphinxAtStartPar
There may be fidgeting movements of the body,

\item {} 
\sphinxAtStartPar
There may be jumping movements of the body,

\item {} 
\sphinxAtStartPar
Arms and legs may be raised or may twitch.

\item {} 
\sphinxAtStartPar
The body may be bent forwards or may recline,

\item {} 
\sphinxAtStartPar
A silvery grey colour may be observed.

\end{enumerate}

\item {} 
\sphinxAtStartPar
PHARANA PĪTI (Pervading rapture)

\sphinxAtStartPar
Characteristics of this Piti include :\sphinxhyphen{}
\begin{enumerate}
\sphinxsetlistlabels{\alph}{enumii}{enumiii}{}{.}%
\item {} 
\sphinxAtStartPar
A feeling of coldness spreads through the body.

\item {} 
\sphinxAtStartPar
Peace of mind sets in occasionally.

\item {} 
\sphinxAtStartPar
There may be itchy feelings all over the body.

\item {} 
\sphinxAtStartPar
There may be drowsy feelings and the meditator may not wish to open his eyes.

\item {} 
\sphinxAtStartPar
The meditator has no wish to move.

\item {} 
\sphinxAtStartPar
There may be a flushing sensation from feet to head or vice versa.

\item {} 
\sphinxAtStartPar
The body may feel cool as if taking a bath or touching ice.

\item {} 
\sphinxAtStartPar
The meditator may see blue or emerald green colours,

\item {} 
\sphinxAtStartPar
An itchy feeling as though lice are crawling on the face may occur.

\end{enumerate}

\end{enumerate}

\sphinxAtStartPar
\sphinxstyleemphasis{This is the end of the description of the five Pitis.}


\subsubsection{PASSADHI}
\label{\detokenize{progress:passadhi}}
\sphinxAtStartPar
The third defilement of Vipassana is PASSADHI which means tranquility or mental factors and consciousness. It is characterized as follows :\sphinxhyphen{}
\begin{enumerate}
\sphinxsetlistlabels{\alph}{enumi}{enumii}{}{.}%
\item {} 
\sphinxAtStartPar
There may be a quiet, peaceful state resembling the attainment of insight,

\item {} 
\sphinxAtStartPar
There will be no restlessness or mental rambling.

\item {} 
\sphinxAtStartPar
Mindful acknowledgement is easy.

\item {} 
\sphinxAtStartPar
The meditator feels comfortably cool and does not fidget.

\item {} 
\sphinxAtStartPar
The meditator feels satisfied with his powers of acknowledgement.

\item {} 
\sphinxAtStartPar
There may be a feeling similar to falling asleep.

\item {} 
\sphinxAtStartPar
There may be a feeling of lightness.

\item {} 
\sphinxAtStartPar
Concentration is good and there is no forgetfulness.

\item {} 
\sphinxAtStartPar
Thoughts are quite clear.

\item {} 
\sphinxAtStartPar
A cruel, harsh or merciless person will realize that the Dhamma is profound. As a result he will give up doing bad and will perform only good actions instead.

\item {} 
\sphinxAtStartPar
A criminal or a drunkard will be able to give up bad habits and will change into quite a different man.

\end{enumerate}


\subsubsection{SUKHA}
\label{\detokenize{progress:sukha}}
\sphinxAtStartPar
The fourth defilement of Vipassana is SUKHA which means bliss and has the following characteristics :\sphinxhyphen{}
\begin{enumerate}
\sphinxsetlistlabels{\alph}{enumi}{enumii}{}{.}%
\item {} 
\sphinxAtStartPar
There may be a feeling of comfort.

\item {} 
\sphinxAtStartPar
Due to pleasant feelings the meditator may not wish to stop but continue practising for a long time.

\item {} 
\sphinxAtStartPar
The meditator may wish to tell other people of the results which he has already gained,

\item {} 
\sphinxAtStartPar
The meditator may feel immeasurably proud and happy.

\item {} 
\sphinxAtStartPar
Some say that they have never known such happiness.

\item {} 
\sphinxAtStartPar
Some feel deeply grateful to their teachers.

\item {} 
\sphinxAtStartPar
Some meditators feel that their teacher is at hand to give help.

\end{enumerate}


\subsubsection{SADDHĀ}
\label{\detokenize{progress:saddha}}
\sphinxAtStartPar
The next defilement of Vipassana is SADDHA which is defined as fervour, resolution or determination and has the following SADDHA characteristics  :\sphinxhyphen{}
\begin{enumerate}
\sphinxsetlistlabels{\alph}{enumi}{enumii}{}{.}%
\item {} 
\sphinxAtStartPar
The practitioner may have too much faith.

\item {} 
\sphinxAtStartPar
He may wish everybody to practice Vipassana.

\item {} 
\sphinxAtStartPar
He may wish to persuade those he comes in contact with to practice.

\item {} 
\sphinxAtStartPar
He may wish to repay the meditation centre for its benefaction.

\item {} 
\sphinxAtStartPar
The meditator may wish to accelerate and deepen his practice.

\item {} 
\sphinxAtStartPar
He may wish to perform meritorious deeds, give alms and build and repair Buddhist buildings and artifacts.

\item {} 
\sphinxAtStartPar
He may feel grateful to the person who persuaded him to practice.

\item {} 
\sphinxAtStartPar
He may wish to give offerings to his teacher.

\item {} 
\sphinxAtStartPar
A meditator may wish to be ordained as a Buddhist monk,

\item {} 
\sphinxAtStartPar
He may not wish to stop practicing.

\item {} 
\sphinxAtStartPar
He might wish to go and stay in a quiet, peaceful place.

\item {} 
\sphinxAtStartPar
The meditator may decide to practice whole\sphinxhyphen{}heartedly.

\end{enumerate}


\subsubsection{PAGGAHA}
\label{\detokenize{progress:paggaha}}
\sphinxAtStartPar
The next defilement of Vipassana is PAGGAHA which means exertion or strenuousness and is defined as follows  :\sphinxhyphen{}
\begin{enumerate}
\sphinxsetlistlabels{\alph}{enumi}{enumii}{}{.}%
\item {} 
\sphinxAtStartPar
Sometimes the meditator may practice too strenuously.

\item {} 
\sphinxAtStartPar
He may intend to practice rigorously, even unto death.

\item {} 
\sphinxAtStartPar
The meditator over exerts himself so that attentiveness and clear\sphinxhyphen{}consciousness are weak causing distraction and lack of Samadhi (concentration)

\end{enumerate}


\subsubsection{UPATTHANA}
\label{\detokenize{progress:upatthana}}
\sphinxAtStartPar
which means mindfulness, is the next defilement of Vipassana to be considered and it is characterized by the following :\sphinxhyphen{}
\begin{enumerate}
\sphinxsetlistlabels{\alph}{enumi}{enumii}{}{.}%
\item {} 
\sphinxAtStartPar
Sometimes excessive concentration upon thought causes the meditator to leave acknowledgement of the present and inclines him to think of the past and the future,

\item {} 
\sphinxAtStartPar
The meditator may be unduly concerned with happenings which took place in the past.

\item {} 
\sphinxAtStartPar
The meditator may have vague recollections of past lives.

\end{enumerate}


\subsubsection{ÑANA}
\label{\detokenize{progress:nana}}
\sphinxAtStartPar
The next (Vipassanupakilesa) to be considered is NANA which means knowledge and is defined as follows :\sphinxhyphen{}
\begin{enumerate}
\sphinxsetlistlabels{\alph}{enumi}{enumii}{}{.}%
\item {} 
\sphinxAtStartPar
Theoretical knowledge may become confused with practice. The meditator misunderstands but thinks that he is right. He may become fond of ostentatiousness and like contending with his teacher.

\item {} 
\sphinxAtStartPar
A meditator may make comments about various objects. For example when the abdomen rises he may say ‘arising’ and when it falls he may say ‘ceasing’.

\item {} 
\sphinxAtStartPar
The meditator may consider various principles which he knows or has studied.

\item {} 
\sphinxAtStartPar
The present cannot be grasped. Usually it is ‘thinking’ which fills up the mind. This may be referred to as ‘thought\sphinxhyphen{}based knowledge.’ Jinta Ñana.

\end{enumerate}


\subsubsection{UPEKKHÂ}
\label{\detokenize{progress:upekkha}}
\sphinxAtStartPar
The ninth defilement of Vipassana is UPEKKHA which has the meaning of not caring or indifference…. It can be defined as follows :\sphinxhyphen{}
\begin{enumerate}
\sphinxsetlistlabels{\alph}{enumi}{enumii}{}{.}%
\item {} 
\sphinxAtStartPar
The mind of the meditator is indifferent, neither pleased or, displeased, nor forgetful. The rising and falling of the abdomen is indistinct and at times imperceptible.

\item {} 
\sphinxAtStartPar
The meditator is unmindful, at times thinking of nothing in particular.

\item {} 
\sphinxAtStartPar
The rising and falling of the abdomen may be intermittently perceptible.

\item {} 
\sphinxAtStartPar
The mind is undisturbed and peaceful.

\item {} 
\sphinxAtStartPar
The meditator is indifferent to bodily needs.

\item {} 
\sphinxAtStartPar
The meditator is unaffected when in contact with either good or bad objects. Mindful acknowledgement is disregarded and attention is allowed to follow exterior objects to a great extent.

\end{enumerate}


\subsubsection{NIKANTI}
\label{\detokenize{progress:nikanti}}
\sphinxAtStartPar
The tenth Vipassanupakilesa is NIKANTI which means ‘gratification’ and it has the following characteristics :\sphinxhyphen{}
\begin{enumerate}
\sphinxsetlistlabels{\alph}{enumi}{enumii}{}{.}%
\item {} 
\sphinxAtStartPar
The meditator finds satisfaction in various objects.

\item {} 
\sphinxAtStartPar
He is satisfied with light, joy, happiness, faith, exertion, knowledge and even\sphinxhyphen{}mindedness.

\item {} 
\sphinxAtStartPar
He is satisfied with various Nimittas (mental images).

\end{enumerate}

\sphinxAtStartPar
\sphinxstyleemphasis{That is the end of the section dealing with the ten Vipassanupakilesas.}


\section{Udayabbaya Ñana}
\label{\detokenize{progress:udayabbaya-nana}}
\sphinxAtStartPar
The fourth ÑANA to be considered is UDAYABBAYA ÑANA which may be translated as knowledge of contemplation on the rise and fall. In this ÑANA the following may occur :\sphinxhyphen{}
\begin{enumerate}
\sphinxsetlistlabels{\alph}{enumi}{enumii}{}{.}%
\item {} 
\sphinxAtStartPar
The meditator sees that the rising and falling of the abdomen consists of 2, 3, 4, 5, or 6 stages.

\item {} 
\sphinxAtStartPar
The rising and falling of the abdomen may disappear intermittently.

\item {} 
\sphinxAtStartPar
Various feelings disappear after two or three acknowledgements.

\item {} 
\sphinxAtStartPar
Acknowledgement is clear and easy.

\item {} 
\sphinxAtStartPar
Nimittas disappear quickly, for instance after a   few acknowledgements of ‘seeing, seeing’.

\item {} 
\sphinxAtStartPar
The meditator may see a clear, bright light.

\item {} 
\sphinxAtStartPar
The beginning and the end of the rising and falling movements of the abdomen are clearly perceived.

\item {} 
\sphinxAtStartPar
While sitting, the body may bend either forwards or backwards as though falling asleep. The extent of the movement depends on the level of concentration. The breaking of ‘Santati’ or continuity can be observed by the expression of the following characteristics :
\begin{enumerate}
\sphinxsetlistlabels{\arabic}{enumii}{enumiii}{}{.}%
\item {} 
\sphinxAtStartPar
If the rising and falling movements of the abdomen become quick and then cease, Anicca (impermanence) appears clearly but Anatta (non\sphinxhyphen{}self) and Dukka (suffering) still continue.

\item {} 
\sphinxAtStartPar
If the rising and falling movements become light and even and then cease, Anatta (non\sphinxhyphen{}self) appears clearly. However Anicca and Dukka still continue.

\item {} 
\sphinxAtStartPar
If the rising and falling of the abdomen becomes stiff and impeded and then ceases, Dukka (suffering) is clearly revealed, but Anicca and Anatta continue.

\end{enumerate}

\end{enumerate}

\sphinxAtStartPar
If the meditator has good concentration, Samadhi, he may experience a ceasing of breath at frequent intervals. He may feel as if he is falling into an abyss or going through an air pocket on a plane, but in fact the body remains motionless.


\section{Bhanga Ñana}
\label{\detokenize{progress:bhanga-nana}}
\sphinxAtStartPar
BHANGA ÑANA is the fifth knowledge or state of wisdom to be considered here. It means ‘Knowledge of contemplation on dissolution’ and it has the following characteristics :\sphinxhyphen{}
\begin{enumerate}
\sphinxsetlistlabels{\alph}{enumi}{enumii}{}{.}%
\item {} 
\sphinxAtStartPar
The ending of the rising and falling movements of the abdomen are clear.

\item {} 
\sphinxAtStartPar
The objects of the meditator’s concentration may not be clear. The rising and falling movements of his obdomen may be vaguely perceived.

\item {} 
\sphinxAtStartPar
The rising and falling movements may disappear. It is however noticed by the practitioner that RUPA disappears first followed by NAMA. In fact the disappearance takes place almost simultaneously because of the swift functioning of the Citta (mind).

\item {} 
\sphinxAtStartPar
The rising and falling movements are distinct and faint.

\item {} 
\sphinxAtStartPar
There is a feeling of tightness enabling one to see the continuity of the rising and falling. The first state of consciousness ceases and a second begins enabling the meditator to know the ceasing.

\item {} 
\sphinxAtStartPar
Acknowledgement is insufficiently clear because its various objects appear to be far away.

\item {} 
\sphinxAtStartPar
At times there is only the rising and falling, the feeling of self disappears.

\item {} 
\sphinxAtStartPar
There may be a feeling of warmth all over the body,

\item {} 
\sphinxAtStartPar
The meditator may feel as though he is covered by a net.

\item {} 
\sphinxAtStartPar
Citta (mind or consciousness) and its object may disappear altogether.

\item {} 
\sphinxAtStartPar
At first RUPA (material or physical) ceases, But Citta remains, however consciousness soon disappears as well as the object of consciousness.

\item {} 
\sphinxAtStartPar
Some meditators feel that the rising and falling of the abdomen ceases for only a short time, while others feel that the movement stops for 2–4 days until they get bored. Walking is the best remedy for this.

\item {} 
\sphinxAtStartPar
Upada, Thiti and Bhanga, that is the coming into being, continuity and passing away stages of both NAMA and RUPA are present but the meditator is not interested, observing only the stage of passing,

\item {} 
\sphinxAtStartPar
The internal objects of meditation, that is the rising falling movements of the abdomen are not clear. External objects, trees etc. seem to shake,

\item {} 
\sphinxAtStartPar
Everything gives the impression of looking at a field of fog, vague and obscure,

\item {} 
\sphinxAtStartPar
If the meditator looks at the sky it seems as it there is vibration in the air.

\item {} 
\sphinxAtStartPar
Rising and falling suddenly ceases and suddenly reappears.

\end{enumerate}


\section{Bhaya Ñana}
\label{\detokenize{progress:bhaya-nana}}
\sphinxAtStartPar
The sixth state of knowledge is BHAYA ÑANA or ‘Knowledge of the appearance as terror’. The following characteristics can be observed :\sphinxhyphen{}
\begin{enumerate}
\sphinxsetlistlabels{\alph}{enumi}{enumii}{}{.}%
\item {} 
\sphinxAtStartPar
At first the meditator acknowledges objects but the acknowledgements vanish together with the consciousness.

\item {} 
\sphinxAtStartPar
A feeling of fear occurs but it is unlike that generated by seeing a ghost.

\item {} 
\sphinxAtStartPar
The disappearance of NAMA and RUPA and the consequent becoming nothing induces fear.

\item {} 
\sphinxAtStartPar
The meditator may feel neuralgic pain similar to that caused by a nervous disease when he is walking or standing.

\item {} 
\sphinxAtStartPar
Some practitioners cry when they think of their friends and relatives.

\item {} 
\sphinxAtStartPar
Some practioners are very much afraid of what they see, even if it is only a water jug or a bed post.

\item {} 
\sphinxAtStartPar
The meditator now realizes that NAMA and RUPA which were previously considered to be good, are completely insubstantial.

\item {} 
\sphinxAtStartPar
There is no feeling of happiness, pleasure or enjoyment.

\item {} 
\sphinxAtStartPar
Some practioners are aware of the feeling of fear but are not controlled by it.

\end{enumerate}


\section{Ādīnava Ñana}
\label{\detokenize{progress:adinava-nana}}
\sphinxAtStartPar
The seventh knowledge, ‘Knowledge of contemplation on disadvantages’ or ĀDĪNAVA ÑANA has the following characteristics :\sphinxhyphen{}
\begin{enumerate}
\sphinxsetlistlabels{\alph}{enumi}{enumii}{}{.}%
\item {} 
\sphinxAtStartPar
The rising and falling movements appear vague and obscure and the movements gradually disappear.

\item {} 
\sphinxAtStartPar
The meditator experiences negative, irritable feelings.

\item {} 
\sphinxAtStartPar
NAMA and RUPA can be acknowledged well.

\item {} 
\sphinxAtStartPar
The meditator is aware of nothing but negativity caused but the arising, continuing and passing away of NAMA and RUPA. The meditator becomes aware of Anicca (impermanence) Dukkha (suffering) and Anatta (non\sphinxhyphen{}self), which are referred to collectively as the TRILAKSANA.

\item {} 
\sphinxAtStartPar
In contrast to former days, acknowledgement of what is perceived by the eyes, nose, tongue, body and mind cannot be made clearly.

\end{enumerate}


\section{Nibbida Ñana}
\label{\detokenize{progress:nibbida-nana}}
\sphinxAtStartPar
NIBBIDA ÑANA or ‘Knowledge of contemplation on dispassion,’ is the eighth NANA. It has the following characteristics :\sphinxhyphen{}
\begin{enumerate}
\sphinxsetlistlabels{\alph}{enumi}{enumii}{}{.}%
\item {} 
\sphinxAtStartPar
The meditator views all objects as tiresome and ugly.

\item {} 
\sphinxAtStartPar
The meditator feels something akin to laziness but the ability to acknowledge objects clearly is still present.

\item {} 
\sphinxAtStartPar
The feeling of joy is absent and the meditator feels bored and sad as though he has been separated from what he loves.

\item {} 
\sphinxAtStartPar
The practitioner may not have experienced boredom before but now he really knows what boredom is.

\item {} 
\sphinxAtStartPar
Although previously the meditator may have thought that only hell was bad, at this stage he feels that only Nibbana, not a heavenly state, is really good. He feels that nothing can compare with Nibbana so he deepens his resolve to search for it.

\item {} 
\sphinxAtStartPar
The meditator may acknowledge that there is nothing pleasant about NĀMA and RUPA.

\item {} 
\sphinxAtStartPar
The meditator may feel that everything is bad in every way and there is nothing that can be enjoyed,

\item {} 
\sphinxAtStartPar
The meditator may not wish to speak to or meet anybody. He may prefer to stay in his room.

\item {} 
\sphinxAtStartPar
The meditator may feel hot and dry as though being scorched by the heat of the sun.

\item {} 
\sphinxAtStartPar
The meditator may feel lonely, sad and apathetic, k. Some lose their attachment to formerly desired fame and fortune. They become bored realizing that all things are subject to decay. All races and beings, even the Devas and Brahmas are likewise subject to decay. They see that where there is birth; old age, sickness and death prevail. So there is no feeling of attachment. Boredom therefore sets in together with a strong inclination to search for Nibbana.

\end{enumerate}


\section{Muncitukamayatā Ñana}
\label{\detokenize{progress:muncitukamayata-nana}}
\sphinxAtStartPar
The ninth Ñana to be considered is MUCITUKAMAYATA NANA which can be translated as ‘The knowledge of the desire for deliverance.’ This Ñana has the following characteristics :\sphinxhyphen{}
\begin{enumerate}
\sphinxsetlistlabels{\alph}{enumi}{enumii}{}{.}%
\item {} 
\sphinxAtStartPar
The meditator itches all over his body. He feels as if he has been bitten by ants or small insects, or he feels as though they are climbing on his face and body.

\item {} 
\sphinxAtStartPar
The meditator becomes impatient and cannot make acknowledgements while standing, sitting, lying down or walking.

\item {} 
\sphinxAtStartPar
He cannot acknowledge other minor actions.

\item {} 
\sphinxAtStartPar
He feels uneasy, restless and bored.

\item {} 
\sphinxAtStartPar
He wishes to get away and give up meditation.

\item {} 
\sphinxAtStartPar
Some meditators think of returning home, because they feel that their Parami (accumulated past merit) has been insufficient. As a result they start preparing their belongings to go home. In the early days this was termed ‘The Ñana of rolling the mat.’

\end{enumerate}


\section{Patisankhā Ñana}
\label{\detokenize{progress:patisankha-nana}}
\sphinxAtStartPar
The tenth ÑANA to be considered here is PATISANKHA ÑANA or the ‘Knowledge of reflective contemplation.’ The following characteristics may be observed :\sphinxhyphen{}
\begin{enumerate}
\sphinxsetlistlabels{\alph}{enumi}{enumii}{}{.}%
\item {} 
\sphinxAtStartPar
The meditator may experience feelings similar to being pierced by splinters throughout his body.

\item {} 
\sphinxAtStartPar
There may be many other disturbing sensations but they disappear after two or three acknowledgements.

\item {} 
\sphinxAtStartPar
The meditator may feel drowsy.

\item {} 
\sphinxAtStartPar
The body may become stiff as if the meditator is entering Phalasamapati (a Vipassana trance) but Citta (mind or consciousness) is still active and the auditory channel is still functioning.

\item {} 
\sphinxAtStartPar
The meditator feels as heavy as stone.

\item {} 
\sphinxAtStartPar
There may be a feeling of heat throughout the body.

\item {} 
\sphinxAtStartPar
He may feel uncomfortable.

\end{enumerate}


\section{Sankhārupekhā Ñana}
\label{\detokenize{progress:sankharupekha-nana}}
\sphinxAtStartPar
‘Knowledge of equanimity regarding all formations’ or SANKHARUPEKHA ÑANA follows. This Ñana has the following characteristics :\sphinxhyphen{}
\begin{enumerate}
\sphinxsetlistlabels{\alph}{enumi}{enumii}{}{.}%
\item {} 
\sphinxAtStartPar
The meditator does not feel frightened or glad, only indifferent. The rising and falling of the abdomen is clearly acknowledged as merely being NAMA and RUPA.

\item {} 
\sphinxAtStartPar
The meditator feels neither happiness nor sadness. His presence of mind and consciousness are clear. NAMA and RUPA are clearly acknowledged.

\item {} 
\sphinxAtStartPar
The meditator can remember and acknowledge without difficulty.

\item {} 
\sphinxAtStartPar
The meditator has good concentration. His mind remains peaceful and smooth for a long time, like a car running on a well paved road. The meditator may feel satisfied and forget the time.

\item {} 
\sphinxAtStartPar
Samadhi (concentration) becomes firm, somewhat like pastry being kneaded by a skilled baker.

\item {} 
\sphinxAtStartPar
Various pains and diseases such as paralysis or nervousness may be cured.

\item {} 
\sphinxAtStartPar
It can be said that the characteristics of this ÑANA are ease and satisfaction. The meditator may forget the time which has been spent during practice. The length of time spent sitting might even be as much as one hour instead of the half hour which was originally intended.

\end{enumerate}


\section{Anuloma Ñana}
\label{\detokenize{progress:anuloma-nana}}
\sphinxAtStartPar
ANULOMA ÑANA or ‘Conformity knowledge’, ‘Adaptation knowledge’ follows. This Ñana can be divided into the following stages :\sphinxhyphen{}
\begin{enumerate}
\sphinxsetlistlabels{\alph}{enumi}{enumii}{}{.}%
\item {} 
\sphinxAtStartPar
Wisdom derived from the preliminary Ñanas starting with the fourth.

\item {} 
\sphinxAtStartPar
Wisdom derived from the higher Ñanas ie. The 37 Bodhipakkiyadharma , qualities contributing to or constituting enlightenment; the 4 Iddhipada or paths of accomplishment; the 4 Sammappadhara, right or perfect efforts; the 4 Satipatthana or foundations of mindfulness; the 5 Indriya or controlling faculties and the five Bhala or powers etc.

\end{enumerate}

\sphinxAtStartPar
Anuloma Ñana has the characteristics of Anicca, Dukkha and Anatta.
\begin{enumerate}
\sphinxsetlistlabels{\arabic}{enumi}{enumii}{}{.}%
\item {} 
\sphinxAtStartPar
Anicca (impermanance) He who has practised charity and kept the precepts will attain the pa’h by Anicca. The rising and falling of the abdomen will become quick but suddenly cease. The meditator is aware of cessation of movement as the abdomen rises and falls or the cessation of sensation when sitting or touching. Quick breathing is a symptom of Anicca, The knowledge of this ceasing whenever it occurs is called Anuloma Ñana. However this should actually be experienced by the meditator, not just imagined.

\item {} 
\sphinxAtStartPar
Dukkha (suffering) He who has practised Samatha (concentration) will attain the path by way of Dukkha. Thus when he acknowledges the rising and falling of the abdomen or sitting and touching, he feels stifled. When he continues to acknowledge the rising and falling of the abdomen or the sitting and touching, a cessation of sensation will take place. A characteristic of path attainment by way of Dukkha is unbearability. The knowledge of the ceasing of the rising and falling of the abdomen, or the cessation of sensation when sitting or touching is Anuloma Nana.

\item {} 
\sphinxAtStartPar
Anatta (No\sphinxhyphen{}self) He who has practised Vipassana or was interested in Vipassana in former lives will attain the path by Anatta. Thus the rising and falling of the abdomen becomes steady, evenly\sphinxhyphen{}spaced and then ceases. The rising and falling movements of the abdomen or the sitting and touching will be seen clearly. Path attainment by Anatta is characterised by a smooth, light movement of the abdomen.

\end{enumerate}

\sphinxAtStartPar
When the movements of the abdomen continue evenly and lightly, that is Anatta. Anatta means ‘without substance’ ‘meaninglessness’ and ‘uncontrollability’.

\sphinxAtStartPar
The ability to know clearly the cessation of the rising and falling movements of the abdomen, or the cessation of sensation when sitting and touching is called Anuloma Ñana.


\subsection{The Four Noble Truths}
\label{\detokenize{progress:the-four-noble-truths}}
\sphinxAtStartPar
In the Anuloma Ñana, the four noble truths appear clearly and distinctly as follows :\sphinxhyphen{}
\begin{enumerate}
\sphinxsetlistlabels{\arabic}{enumi}{enumii}{}{.}%
\item {} 
\sphinxAtStartPar
SAMUDAYA SACCA. This truth is perceived when the abdomen begins to rise or begins to fall, and it occurs at the point that the meditator is about to enter the next Nana, which is called the Gotrabhu Ñana. Samudaya Sacea is also referred to as Rupa Jati and Nama Jati. It is the point of origination of Nama and Rupa. It is the point of origination of both the beginning of the rising and the beginning of the falling movements of the abdomen. Nama Jati is the beginning of Nama and Rupa Jati is the beginning of Rupa. Real perception and experience of these truths is called ‘Samudaya Sacca’.

\item {} 
\sphinxAtStartPar
DUKKA SACCA. This truth is perceived when the rising and falling movements of the abdomen can no longer be tolerated because the meditator is aware of their unsatisfactory nature. He perceives that everything must die out and come to an end. In Pali this truth is given the name CHARĀMARANAM DUKKHA SACCAM. Old age is a deterioration of Nama and Rupa. Death is the extinction, the breaking up, the ending of Nama and Rupa. Death is the extinction, the breaking up, the ending of Nama and Rupa. The perception of the cessation of suffering is called Dukka Sacca.

\item {} 
\sphinxAtStartPar
NIRODHA SACCA. This truth is seen when the rising and falling movements fall away simultaneously. Jati is the limit of knowledge and so the mental acknowledgement of the cessation of the movements of the abdomen also fades away at the same time. This constitutes the state of Nibbana. In Pali this is referred to as ‘Ubhinnampi Nissarnam, The state when Dukka (suffering) and the point of origination of Nama Rupa (Samudaya) both cease is called ‘Nirodha Sacca’.

\item {} 
\sphinxAtStartPar
MAGGA SACCA. (The Great Truth) In this state of knowledge or wisdom, the meditator is completely aware of the rising and falling of the abdomen. He is aware of the beginning of the rising and falling the middle of the rising and falling and the points when the rising and falling cease. In Pali this state is known, as ‘NIROTHAPPACHANANA MAGGA SACCAM. When the ending of suffering and the cessation of the movements of the abdomen are clearly seen this is termed Magga Sacca.

\end{enumerate}

\sphinxAtStartPar
It is necessary for the practitioner to be aware of these four truths simultaneously. It should be like blowing out a candle, ie.
\begin{enumerate}
\sphinxsetlistlabels{\arabic}{enumi}{enumii}{}{.}%
\item {} 
\sphinxAtStartPar
It should be like the point at which the wick of the candle has been used up.

\item {} 
\sphinxAtStartPar
It should be like the point at which the wax of the candle has been used up.

\item {} 
\sphinxAtStartPar
It should be like an overwhelming brilliance which has obliterated the candle light.

\item {} 
\sphinxAtStartPar
It should be like a deep darkness.

\end{enumerate}

\sphinxAtStartPar
The four characteristics of the light given here are likely to appear at the same time and at the same level as the perception of the Four Noble Truths. The state of Nibbana is perceived in Nirodha Sacca, Dukka Sacca Samudaya Sacca and Magga Sacca at the same time.


\section{Gotrabhu Ñana}
\label{\detokenize{progress:gotrabhu-nana}}
\sphinxAtStartPar
The next Ñana to be considered is GOTRABHU ÑANA or ‘Knowledge at the moment of change of lineage’. Gotrabhu Ñana is the knowledge which entirely separates one from the worldly state. Nama and Rupa, together with Citta (mind) which has become aware of the cessation, both become peaceful and quiet. This means one has become enlightened, having Nibbana as the object. The moment when feeling breaks off, Gotrabhu Ñana is reached.
\begin{enumerate}
\sphinxsetlistlabels{\arabic}{enumi}{enumii}{}{.}%
\item {} 
\sphinxAtStartPar
Uppadam abhibhuyyatiti gotrabhu : Knowledge which covers the arising of Nama and Rupa is called Gotrabhu.

\item {} 
\sphinxAtStartPar
Pavattam abhibhuyyatiti gotrabhu : Knowledge which covers the continuance of Nama and Rupa is called Gotrabhu.

\item {} 
\sphinxAtStartPar
Bahiddhasamkhanranirnittam abhibhuyyatiti gotrabhu: Knowledge which covers the external Nama and Rupa is called Gotrabhu.

\item {} 
\sphinxAtStartPar
Anuppadam Pakkhandatiti gotrabhu : Knowledge which moves towards cessation is called Gotrabhu.

\item {} 
\sphinxAtStartPar
Appavattam nirodham nibbaham pakkhandhatiti gotrabhu : Knowledge which approaches the discontinuance, cessation and Nibbana is called Gotrabhu.

\item {} 
\sphinxAtStartPar
Uppadam abhihuyyatva anuppadam pakkhandatiti gotrabhu : Wisdom which covers the arising and then approaches the non\sphinxhyphen{}arising is called Gotrabhu.

\end{enumerate}

\sphinxAtStartPar
To summarize, the moment that feeling breaks off the first time, for about half a minute is called ‘Gotrabhu Nana’. The meditator casts off Ñama and Rupa. Awareness grasps Nibbana as its object. This state is between LOKIYA (worldly existance) and LOKUTTARA (supramundane existance). It is not a state of worldly existance or a state of supramundane existance, because it is in between both states. It is like a man who enters a Buddha’s hall, one of his feet is outside and the other is inside. You cannot say that he is outside or inside.


\section{Magga Ñana}
\label{\detokenize{progress:magga-nana}}
\sphinxAtStartPar
The next Ñana to be considered is MAGGA ÑANA. It can be translated as ‘Knowledge of the path’. In this Nana, defilements have been broken off (Samucchedpahara) MAGGA ÑANA has the following characteristics :\sphinxhyphen{}
\begin{enumerate}
\sphinxsetlistlabels{\arabic}{enumi}{enumii}{}{.}%
\item {} 
\sphinxAtStartPar
The destruction of some defilements and preparation for the destruction of others. It constitutes a cleansing.

\item {} 
\sphinxAtStartPar
There is clear and complete knowledge of path.

\item {} 
\sphinxAtStartPar
There is a deep knowledge of Dhamma which leads to Nibbana.

\item {} 
\sphinxAtStartPar
Magga Ñana is a deep knowledge of Dhamma which is necessary to reach Nibbana.

\item {} 
\sphinxAtStartPar
It is a deep wisdom which enables the practitioner to eradicate defilements.

\end{enumerate}

\sphinxAtStartPar
Characteristics of Magga Ñana are :\sphinxhyphen{}
\begin{enumerate}
\sphinxsetlistlabels{\arabic}{enumi}{enumii}{}{.}%
\item {} 
\sphinxAtStartPar
After the breaking off of sensation, awareness of the stream of Nibbana lasts for a moment. Some defilements are completely destroyed. Sense of self (ego), sceptical doubt, and a misunderstanding of rules and rituals and diciplines will be cut off during this Ñana. This Ñana has Nibbhana as its object. Nibbana can be reached. There is no doubt about what is right and wrong, about heaven and hell, about the path, the result of the path and Nibbana. There is no doubt concerning life after death. This Ñana is supramundane.

\item {} 
\sphinxAtStartPar
Anuloma Ñana is the last Ñana in which there is happening. After that there is no awareness of anything. Feeling and awareness suddenly cease. It is like a person who is walking along a road and suddenly falls down a hole. The object and the mind which is trying to acknowledge the object both cease to function is the state of Nibbana. This cessation is called Gotrabhu Ñana. This state of wisdom encompasses the cessation of awareness and form.

\item {} 
\sphinxAtStartPar
After Gotrabhu Ñana has lasted a moment this is termed MAGGA ÑANA.

\end{enumerate}


\section{Phala Ñana}
\label{\detokenize{progress:phala-nana}}
\sphinxAtStartPar
The fifteenth Nana is called PHALA ÑANA or the ‘Knowledge of fruition’. This occurs a moment after Magga ñana. The mind has come to know what happened and has Nibbana as the object. This state lasts for two or three moments. Whenever Magga Ñana happens Phala Ñana follows immediately. There is no interim state. Phala Ñana like Magga Ñana is supramundane. Magga Ñana is the cause and Phala Ñana is the result. The way of entering Gotrabhu Ñana, Magga Ñana and Phala Ñana is as follows :\sphinxhyphen{}
\begin{enumerate}
\sphinxsetlistlabels{\arabic}{enumi}{enumii}{}{.}%
\item {} 
\sphinxAtStartPar
The first cessation of sensation is Gotrabhu Ñana and it has Nibbana as its object. It lies between the mundane and the supramundane existances.

\item {} 
\sphinxAtStartPar
The midway cessation of sensation is Magga Ñana and it has Nibbana as its object. It is supramundane. At this point defilements are eradicated.

\item {} 
\sphinxAtStartPar
The final cessation is called Phala Ñana and it has Nibbana as its object. It is also supramundane. The eradication of defilements of Magga Ñana is called ‘Samucchedpahara’ and means the complete eradication of defilements. In Phala Nana those defilements are prevented from re\sphinxhyphen{}occuring This lack of re\sphinxhyphen{}occurance is termed ‘Patipasamphana Pahara’ in Phala Nana. This process may be compared to extinguishing a fire. Imagine a piece of wood which is on fire. If you want to put the fire out you must throw water on the wood so that the flames die down, but the wood will continue smouldering. However if the wood is doused with water again two or three times, the fire will be completely extinguished. This parallels what happens when a meditator eradicates defilements during Magga ñana. The power of defilements still continues so it is necessary to purge it again during Phala Ñana. (Patipasamphana Pahara) is like the second and third applications of water to put out the fires of defilements.

\end{enumerate}


\section{Paccavekkhana Ñana}
\label{\detokenize{progress:paccavekkhana-nana}}
\sphinxAtStartPar
The sixteenth Ñana is called PACCAVEKKHANA ÑANA or ‘Knowledge of Reviewing.’ In this Ñana there is a knowledge and contemplation of the path, the fruit, and Nibbana. There is a knowledge of those defilements which have been eradicated and those which still continue.
\begin{enumerate}
\sphinxsetlistlabels{\arabic}{enumi}{enumii}{}{.}%
\item {} 
\sphinxAtStartPar
There is a contemplation of having followed the path.

\item {} 
\sphinxAtStartPar
There is a contemplation of the fact that a result has been obtained.

\item {} 
\sphinxAtStartPar
There is a contemplation of the defilements which have been eradicated.

\item {} 
\sphinxAtStartPar
There is a contemplation of the defilements which remain.

\item {} 
\sphinxAtStartPar
There is a contemplation of the fact that Nibbana which is an exceptional state of awareness has been known and experienced.

\end{enumerate}

\sphinxAtStartPar
In addition while the meditator is acknowledging rising and falling, he comes upon the path, the fruit and Nibbana. At the moment he enters the path, the fruit and Nibbana, three conditions occur : Anicca, Dukka and Anatta as previously mentioned. Paccavekkhana Ñana means when the meditator is acknowledging the rising and falling motions of the abdomen he is aware of the total cessation of the rising and falling. After the cessation, when awareness returns, the meditator contemplates what has happened to him. After this he goes on acknowledging the rising and falling movements but they seem much clearer than normal. Considering what has happened is called ‘Paccavekkhana Ñana’.


\section{SAMAPATI or Vipassana Trance}
\label{\detokenize{progress:samapati-or-vipassana-trance}}
\sphinxAtStartPar
There are three kinds of Samapati; Chanasamapati. Phalasamapati and Nirodhasamapati. Here Phalasamapati only will be discussed.

\sphinxAtStartPar
Phalasamapati means seeing the result of the path which you have gained. You can make a wish to enter a Vipassana trance for 5 minutes, 10 minutes, 24 hours or longer. The length of time depends on the power of your concentration. If you have good concentration you can stay in the trance state for a long time, but if your concentration is not good you can stay in the trance state for just a short time.

\appendix
% move PDF bookmarks to the top level
\bookmarksetup{startatroot}
% demote sections again, same as in frontmatter
%\let\part\chapter
%\let\chapter\section
%\let\section\subsection
%\let\subsection\subsubsection

%\makeatletter
%\patchcmd{\@chapter}{\addcontentsline{toc}{chapter}}{\addcontentsline{toc}{part}}{}{}
%\patchcmd{\@chapter}{\addcontentsline{toc}{chapter}}{\addcontentsline{toc}{part}}{}{}
%\makeatother

\sphinxstepscope


\chapter{An appreciation of Phra Dhamma Theerarach Mahamuni and his teachings}
\label{\detokenize{appreciation:an-appreciation-of-phra-dhamma-theerarach-mahamuni-and-his-teachings}}\label{\detokenize{appreciation::doc}}
\sphinxAtStartPar
\sphinxstyleemphasis{(by George D. Bickell)}

\sphinxAtStartPar
I first met Phra Dhamma Theerarach Mahamuni in the year 2507 (1964 C.E.), on his first journey to Britain. His first engagement was to participate in the opening of the Sri Lankan Wat and Vihara at Chiswick, his second, to teach at the Viharn in Haverstock Hill.

\sphinxAtStartPar
There was always a large audience at this venue. It was well situated and the charisma of the incumbent Phra Ananda Bodhi was much appreciated by the people that lived in the locale and also by many professed Buddhists, these already knew that’ Buddhism was something to do, not something to believe in, not something just to be told about, not something to just talk about.

\sphinxAtStartPar
The audience were not to be disappointed, in his characteristic manner he immediately started to lecture on the practicalities of purification of knowledge and vision. However some were disappointed for on this occasion the building was overcrowded, so much so, that some people had to be turned away. The fame of Phra Rajasiddhimuni “as Phra Dhamma Theerarach Mahamuni then was”, had ensured this.

\sphinxAtStartPar
This meeting and the others that followed it were a great success, at all hours there was a queue of people on the staircase waiting to give him reports on their meditation experiences, again people had to be turned away lest he be deprived of any opportunity to sleep and rest.

\sphinxAtStartPar
For persons such as myself who had realised the inadequacy of western approaches to an understanding of mind and its problems, its effects on individuals, relationships and society and who had in consequence been investigating eastern understanding such as Yoga, Hinduism, mysticism and their associated Samatha Techniques, struggling with a limited and \textasciicircum{}vondhand understanding of true Dhamma; his demonstrations as to what was path and not path were mind bending and revealing.

\sphinxAtStartPar
This was what many western people had been seeking, for there was and is a large audience seeking an effective method to overcome their frustrations and fears. For although there were one or two books available on the subject of vipassana meditation, there had been no teacher available to demonstrate the actions of the hindrances and defilements of meditation and the meaningfulness of these. For no book can be as pointed and as revealing as a good teacher.

\sphinxAtStartPar
What attracted people to him was his energy, confidence and enthusiasm. He had the ability to express the most abstract Dhamma in a simple and direct manner, this coupled with his immediacy in comprehending the mental states of his many pupils gave those around him confidence too and this confidence was transformed into hope, enthusiasm and expectancy when he declared the possibilities of attainment of better states of mind and even Nirodha Sacca through the practise of concentrated mindfulness for all, young, old, rich and poor, monk and layman alike and then verbally and physically demonstrated the method.

\sphinxAtStartPar
This joy in teaching remained with him to the end of his days and this witness to the above events, still stood in awe and marvelled when twenty years later as a monk in section 5, he watched this great teacher still instructing young and old alike with the same enthusiasm as he had in former times.

\sphinxAtStartPar
The ability to teach and explain and the great pleasure that he displayed when he was doing this, was the result of his own experience, for although he was a great scholar he had transformed his learning into direct experience and was able to transform his experience into simple terminology supported by a deep and profound practical knowledge of the Pali Canon and he knew that with the right guidance others could also experience and understand the same knowing and the number of his successful pupils testifies to the truth of this.

\sphinxAtStartPar
These were great and glorious days for the Dhamma in Britain and whilst much has happened to Buddhism and Buddhists in Britain since, “some good, some bad!”, his influence can still be seen there. For when one studies the situation in Britain now, names from that period can still be found that are continuing to spread that which they learned from him.

\sphinxAtStartPar
The following year Wat Buddhapadipa opened its doors to the public, staffed by monks trained by him and the teaching was established in a more orderly manner with Phra Dhamma Theerarach Mahamuni travelling yearly to this and other centres in Europe, guiding and inspiring those that he met, this he continued to do for a number of years until his many duties in Thailand made this impossible but even then old friends and new potential students were always welcomed at section 5 and could and did benefit directly and personally from his vast experience.

\sphinxAtStartPar
Popes, Abbots, Prime Ministers and other office holders can always be replaced but persons and especially persons of such calibre as Phra Dhamma Theerarach Mahamuni are unique and we cannot hope to see such as he again, our only hope, our only chance for a better understanding of our own situation, a chance for a better world for all, is that we try to do what he did. It is our responsibility to put the best, the most effective teachings that we know into practise and we can know no better teachings than those that we have received from The Buddha as expressed and explained by Phra Dhamma Theerarach Mahamuni and we must try and convert his words and thought into our actions.

\sphinxAtStartPar
The beauty of the teaching as expressed by Phra Dhamma Theerarach Mahamuni is that it crosses all cultural barriers, this means both those barriers within a culture and those barriers between cultures. Schooling or a lack of it is not a barrier either.

\sphinxAtStartPar
Many foreigners who have practised meditation under my guidance at section 5 can testify to this, for every person seems to believe that they have special problems and are unique, they are therefore amazed when they find that the meditation experience penetrates their nationality and class and demonstrates their humanity, the humanity that we all share.

\sphinxAtStartPar
Some such yogins have compared the meditation experience of a few days with months and even years spent on a psychotherapists couch and have been surprised that such meaningful experiences had been denied them for such a long time and that their particular problems, misunderstandings and fears were common to all, but that few had recognised this or had seen that a solution was possible.

\sphinxAtStartPar
There are difficulties of course and these are caused by the simplicity of the method, it is very simple but persons because of their schooling and conditioning tend to be, to want and to make everything very complicated and at first it would seem to be very difficult to understand that the experience of a condition is a different experience from that of knowing a condition. However the practice will quickly demonstrate this simple truth, one will’ easily see that knowing that one is angry is a condition that can replace anger.

\sphinxAtStartPar
It is important that the method is initially learned by a period of intensive practice, to fully understand it and to prevent errors in emphasis and perception arising and therefore a good instructor, guide and the right conditions are needed, then the system is seen as a most practical one as it can and should be applied to all aspects of daily life, whether in the home, a factory or an office and it is therefore an ideal method for those that cannot or are not prepared because of family or social commitments, to retire from the world.

\sphinxAtStartPar
This daily and consistent practice alters the quality of life of those that follow it, their minds are alive, alert and not contaminated in the same way as the mind of someone caught up in the usual whirl of worldly conditions and they are therefore in a position to experience insights denied to others. Such practise keeps the mind clean, uses less energy and therefore can make one more efficient in the ordering of mundane matters, this cleanliness of mind is a useful platform from which to enter a meditation centre for a further and perhaps more meaningful period of retreat.

\sphinxAtStartPar
There have also been criticisms, one such being that the method involves unusual behaviour and is therefore unnatural, a brief answer to this would be that if “normal” behaviour has made beings into that which they are, greedy, full of prejudice and delusion the cure must lie in a different mode of behaviour. Certainly the slow walking technique is a contradiction to the usual frenetic behaviour that people engage in and will prove how difficult it is to relax in this age of compulsive competition and aggressiveness.

\sphinxAtStartPar
Other critics having not realised the momentary nature of the mind and its innate manner of functioning have based their criticism on their own thinly\sphinxhyphen{}disguised but not understood need for a soul theory. These persons endeavour to substitute a watcher for the watching and also try to prove that this phantasm, “the watcher” is being distorted by the practice.

\sphinxAtStartPar
Some critics have also suggested that the naming or labelling of phenomena is not necessary and is an introduction of mental activity that is unnatural or conceptual, such persons are probably suffering from the ignorance of book information which they believe to be knowledge, not understanding that they are the victims of concepts which are contradictions of knowledge that they could have.

\sphinxAtStartPar
The naming process when done correctly replaces the usual mental activity which is based on greed, aversion and the delusions that the impermanent is permanent, that the ugly is beautiful and that phenomena is uncaused or that there is beautiful and that phenomena is uncaused or that there is a self performing or responsible for the mental and physical phenomena occurring. Even when the mental activity is obsessive and does not respond easily to mindfulness and energy, the naming of the phenomena is at the very least a temporary interuption of this obsessiveness, the naming also reduces the effect of clinging and craving.

\sphinxAtStartPar
Furthermore quietening the mind in this way helps to give the clear comprehension mentioned in Sutta that is needful for clear understanding, if such critics had a viable alternative method then all the preceeding information would be obvious to them and they could not indulge in their verbal inanities.

\sphinxAtStartPar
Some of this criticism comes from persons that have tried the practice and have not succeeded in understanding that the mind has a natural resistance to understanding itself, to exposing itself to itself, that it has a natural urge to be at peace with itself even if this peace involves being ignorant of the truth of the determining factors of the attributes of the mind. Beginning to become aware of their true nature, their greeds, their obsessions, their aversions, their prejudices, their weaknesses they run away from the experience and then condemn the method that starts to expose thus their minds.

\sphinxAtStartPar
One of the problems of psychology is that it only serves cultures it does not serve human beings, the psychologists of each country try to fit the human person into a particular form of society, but the form of every society is ignorance, igorance of the three signs of being and ignorance of the nature of mind and its relationships to matter, or the relationship of mind and mind unless we are to talk about the adepts of advertising who with their ignorance know how to instill craving and hysteria into beings.

\sphinxAtStartPar
All phobia and other recognised mental disorders that are recognized as being antisocial are an over emphasis on a sense of “I” and the western approach to such problems is to attempt to reconcile the patient to his condition, to show the patient causes for the condition and suggest that this way of seeing things changes things, another approach is just to sedate the patient with drugs not words.

\sphinxAtStartPar
Neither approach can work, for neither the psychiatrist, the psychologist nor the psychotherapist understand that the sense of “I” that is the root cause of these apparent disorders is one that they subscribe to themselves, nor do they realise the distortion that this wrong view can give to perception, or have any experience except in the field of psychosomatics, that mind can and does modify body activity and function.

\sphinxAtStartPar
Ignorance of the four noble truths and the three signs of being is the cause of difficulties in relationships that are so common in the west and also for the non acceptance of the status that one has, and it also makes understanding and integrating the various facets of one’s personality very difficult and alienation so very common. Practical insight meditation is the only solution to this condition of ignorance.

\sphinxAtStartPar
The practice applied methodically will show the fleeting nature of thought and the compulsive nature of craving and clinging and demonstrate how obsessions arise and then dominate bodily action, determine the quality and nature of speech and colour all the subsequent mental activity, and how this mental activity governs perception and how this perception again causes and controls mental and physical activity, which again affects perception, thus the endless round of distortion is seen. But this seeing changes the determining factors, for it is a new experience and as such a modifying one.

\sphinxAtStartPar
Pyschosomatic malfunctions of the body can and often do cease under this discipline. Mental trauma can also be cleared, sometimes this can happen by a reliving and revision of the original traumatic experience, sometimes the problem manifests and is cleared in symbolic terms and as such freedom from a past experience can be gained with or without a knowledge of the unpleasantness that has been lost and as such is a deconditioning experience.

\sphinxAtStartPar
Our ignorance of our motivations oppresses others, we manifest it as a selfishness that we label as caring, this selfishness dominates our relationships and is usually encouraged by the demands of family, friends and society, individuals must suffer as a consequence of these attitudes.

\sphinxAtStartPar
Kindness, caring, responsiveness is not kindness, caring and responsiveness if it excludes any person, if it is not available to all, if it is selective and only applied to, my relatives, my friends, my conutrymen, my race, or some other manifestation of “my”, then it is not giving, it is not love, it is not caring, it is selfishness, it is greed, it is pride, it is not the result of right views and right effort.

\sphinxAtStartPar
This manner of mental activity makes our worlds small, very small and very vulnerable, such small worlds have no support no foundation for they rest on an illusion of permanence, of self, of separateness and the idea that things can be controlled. These worlds are worlds of suffering and we are this suffering.

\sphinxAtStartPar
We should remember that all these ideas of relatives, friends, countrymen and even races are, as mental activities, subject to cessation and we should also remember that the objects of these activities and perceptions are mortal and as such are a multiple \sphinxhyphen{}source of suffering.

\sphinxAtStartPar
If our worlds are dependent upon such greed, such exclusiveness, then they are indeed very small and easily shaken, only when we have an all embracing giving, an all embracing caring and an all embracing compassion will we be free of our past, when we can give and care with no distinction as to class, caste or other background of discrimination, only then can we be truly free. For then our worlds will have become truly universal, will not be limited by our greeds and as such have a reality and not be the product of our delusions.

\sphinxAtStartPar
This the practice can teach us and it will teach us more, it can give us practice in living in the here and now, the experience of spontaneity. It can give us practice in the seeing of mind as mind and body as body. It will show us why we need to practise, it leads to cessation and the fruit of that cessation but we must make the effort, we must do the work, we must learn to see, even great men such as Phra Dhamma Theerarach Mahamuni can only guide us.

\sphinxAtStartPar
The practice of this simple technique under the guidance of a competent watcher and guide will quickly demonstrate the fickleness of mind, its aberations, its superficiality and the various trauma that are its foundation. Then one will go on to understand the difference between mind and matter, path and not path and then proceed through all that one needs to know to reach the goal of dhamma, Nibbana, the only meaningful event that living can have.

\sphinxAtStartPar
The effort cannot be wasted, one cannot be a failure, for even if in the unlikely event that the practice is unsuccessful insofar that one does not reach the final goal of Dhamma, worthwhile achievements will be made, for the greatness of our teacher and his teaching is not only an intellectual one but a practical one, for practice of the path as advocated by him leads to caring and involvement. The teaching develops unselfishness, energy, time and mental capabilities for the welfare of others, for the mental and physical wellbeing of all we encounter.

\sphinxAtStartPar
For in such a crowded world as we live in, it is important that we endeavour to give to each of us, not only physical space but also mental space and it is when we can help others to progress on the path that we can have a chance to make realistic progress for ourselves. Self interest has never been a part of this teaching for such a self is a delusion.

\sphinxAtStartPar
The gain of others is a gain for each and everyone of us, it is our responsibility to help ourselves by helping others, our resposibility to improve our world by improving the world of all others, giving smiles, small kindnesses, gifts of material comfort and the greatest of all gifts, “Dhamma”. We need these in our worlds and that is why we must be able to offer them to others.

\sphinxAtStartPar
There have been and are other teachers of vipassana. Some know some of the words, these can be described as labourers. Some understand some of the words, these can be described as craftsmen. Phra Dhamma Theerarach Mahamuni knew and understood the experience, he was an artist and he expected us to become artists too.

\sphinxstepscope


\chapter{Announcement of Passing Away}
\label{\detokenize{death:announcement-of-passing-away}}\label{\detokenize{death::doc}}
\sphinxAtStartPar
It is with great sorrow that
\begin{itemize}
\item {} 
\sphinxAtStartPar
Wat Mahadhatu

\item {} 
\sphinxAtStartPar
Mahachulalongkorn University

\item {} 
\sphinxAtStartPar
the Thai Buddhist Sangha

\item {} 
\sphinxAtStartPar
the Buddhist Sanghas throughout

\item {} 
\sphinxAtStartPar
the world

\item {} 
\sphinxAtStartPar
and all good sentient beings in all the varied planes of existence

\end{itemize}

\sphinxAtStartPar
announce the passing away of \sphinxstylestrong{His Holiness Phra Dhamma Theerarach Mahamuni} on 30 June 2531 in the late afternoon.


\bigskip\hrule\bigskip


\sphinxAtStartPar
In the current history of Thailand there has been no one to compare to the heroic effort, experience, and knowledge of the Tipitaka combined in His Holiness Phra Dhamma Theerarach Mahamuni and his success in bringing Vipassana back as the central concern of Thai Buddhists where it belongs. The loss of such a hero and great lover of the Buddha, Dhamma, and Sangha is one that is mourned deeply by all. May all the good disciples of His Holiness Phra Dhamma Theerarach Mahamuni be quick to remember his fine lessons, practise them well, and pass them on to all those who have groped in the darkness of ignorance, so that others may realize the path of disciplined study and diligent practice that His Holiness Phra Dhamma Theerarach Mahamuni taught, in accordance with the Buddha’s Teachings, for the benefit of all.

\begin{DUlineblock}{0em}
\item[] James P.S. Ross
\item[] U.S.A.
\end{DUlineblock}



\renewcommand{\indexname}{Index}
\printindex
\end{document}